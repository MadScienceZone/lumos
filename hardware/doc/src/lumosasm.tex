\section*{NAME}
lumosasm -- Lumos SSR controller sequence assembler
\section*{SYNOPSIS}
\codetype{lumosasm}
[\codetype{--hHv}]
[\codetype{--l}
\Var*{file}]
[\codetype{--o}
\Var*{file}]
\Var*{asm-files}\dots 
\section*{DESCRIPTION}


Lumos stored sequences may be written in a high-level programming
language (but see the note under ``BUGS'' below) and loaded
into the controller board via
\codetype{lumosctl}(1).
Alternatively, they may be written in the low-level assembly language
which corresponds directly to the bytecode interpreted by the 
controller itself.  
The
\codetype{lumosasm}
command is used for the latter use case.  It converts the sequences
described in one or more 
\Var*{asm-files}
to binary bytecode which is written to a single output file.
\section*{OPTIONS}


Each of the following options may be specified by either a long
option (like
``\codetype{----verbose}'')
or a shorter option letter (like
``\codetype{--v}'').
If an option takes a parameter, it may follow the option as
``\codetype{--lfoo.txt}'',
``\codetype{--l foo.txt}'',
``\codetype{----listing foo.txt}'',
or
``\codetype{----listing=foo.txt}''.


Long option names may be abbreviated to any unambiguous initial substring.
\begin{list}{}{}
% tagged paragraph, width=16
\item[{\codetype{----help}}]\hfill\\
(\codetype{--h})
Prints a summary of these options and exits.
% tagged paragraph, width=4
\item[{\codetype{----list--opcodes}}]\hfill\\
(\codetype{--H})
Prints out a list of all opcodes and the instruction signature they
represent, then exits.
% tagged paragraph, width=4
\item[{\codetype{----listing=}\Var*{file}}]\hfill\\
(\codetype{--l}
\Var*{file})
Causes a listing to be written to the specified 
\Var*{file}.
This will include the assembly code input and byte code output.
By default, no listing is made.
% tagged paragraph, width=4
\item[{\codetype{----output=}\Var*{file}}]\hfill\\
(\codetype{--o}
\Var*{file})
Specifies the name of the output file into which the assembled
binary bytecodes will be written.  By default, this is the name of
the first input file with its suffix replaced by 
``\codetype{.hex}''.
% tagged paragraph, width=4
\item[{\codetype{----verbose}}]\hfill\\
(\codetype{--v})
Each 
\codetype{----verbose}
option increases the level of information generated by the
\codetype{lumosasm}
command.
% tagged paragraph, width=4
\item[{\codetype{----version}}]\hfill\\
Prints the version of this program and exits.
\end{list}
\section*{AUTHOR}


Software Alchemy / support@alchemy.com
\section*{COMPATIBILITY}


This version of 
\codetype{lumosasm}
is compatible with the following boards:
\begin{itemize}
% tagged paragraph, width=2
\item
Lumos 48-channel controller version 3.1 or 3.2
\emph{providing it has been upgraded or installed with ROM firmware version 3.0 or later}
(boards with ID markings beginning with ``48CTL-3-1'' or ``LUMOS-48CTL-3.2'').  (Whether this controller is
driving AC or DC boards is irrelevant.)
% tagged paragraph, width=4
\item
Lumos 24-channel DC controller version 1.0 (boards with ID markings beginning with
``LUMOS-24SSR-DC-1.0'').
\end{itemize}
\section*{HISTORY}


This program is new as of Lumos ROM version 3.0.
\section*{SEE ALSO}
\codetype{lumosctl}(1).


\emph{Using the Lumos SSR Controllers,}
which explains in detail the programming model and assembly language instructions
used with this assembler.
\section*{BUGS}


Currently, the high-level language compiler has yet to be written, so this assembly
code is the only option which is available for writing sequences.  The high-level
compiler is a future task being planned.


Submit any other issues found to 
\codetype{support@alchemy.com}.
