\LLstart{W}{ITH}{all the parts} and tools at hand as described in the previous chapter, you
are now ready to begin assembly of the Lumos controller PC board.  The order of installation
presented here is intended to make assembly as convenient as possible.  Generally this means
progressing from the shortest to the tallest components, allowing the board to be laid flat
face-down on the work surface while soldering the component leads.

\begin{itemize}
\item[\HandRight] \bfseries{Note:} 
Take care to make good, solid \ix{solder connections} when installing components.
Hold your soldering iron to the part's lead \emph{and} the \ix{annular ring} of the \acronym{PCB}
until both are hot, then apply just enough solder to cover the ring, withdraw the solder, then 
remove the heat.  Good solder connections should be shiny and smooth.

\item[\HandRight] \bfseries{Note:} 
The board layout is quite compact, with many components in a small space.  Take
care when soldering that you don't accidentally heat the wrong component or form solder bridges between
nearby contact points.

\item[\HandRight] \bfseries{Note:} 
We will point out a few places where \acronym{ESD} protection is needed, but
that is intended to call attention to the issue at some key points, not to be a comprehensive list of
\emph{every} case where it is needed.  You are expected to use appropriate handling protocols for all
parts, which includes the use of \acronym{ESD} protection when working with semiconductors (e.g., all
chips, voltage regulators, transistors, etc.).
\end{itemize}
