\section*{NAME}
lumosctl -- Manual control for Lumos SSR controller hardware
\section*{SYNOPSIS}
\codetype{lumosctl}
[\codetype{--dhkPRSvwXz}]
[\codetype{--a}
\Var*{addr}]
[\codetype{--A}
\Var*{addr}]
[\codetype{--b}
\Var*{speed}]
[\codetype{--B}
\Var*{speed}]
[\codetype{--c}
\Var*{file}]
[\codetype{--C}
\Var*{file}]
[\codetype{--D}
\Var*{sens}]
[\codetype{--E}
\Var*{sens}]
[\codetype{--H}
\Var*{hexfile}]
[\codetype{--p}
\Var*{port}]
[\codetype{--P}
\Var*{phase}]
[\codetype{--s}
\Var*{file}]
[\codetype{--t}
\Var*{s}\codetype{\textnormal{[}orw+\textnormal{]}:}\Var*{init}\codetype{:}\Var*{seq}\codetype{:}\Var*{term\textnormal{]}}
\Var*{channel-outputs}\dots 


Where
\Var*{channel-outputs}
may be any combination of:
\begin{center}


\Var*{channel}[\codetype{@}\Var*{level}[\codetype{,}\dots ]]
\\
\Var*{channel}\codetype{d}[\Var*{\codetype{:}steps}[\Var*{\codetype{:}time}]]
\\
\Var*{channel}\codetype{u}[\Var*{\codetype{:}steps}[\Var*{\codetype{:}time}]]
\\
\codetype{x}\Var*{id}
\\
\codetype{p}\Var*{time}
\end{center}
\section*{DESCRIPTION}


This command allows you to directly manipulate the state of a
supported Lumos
SSR controller unit, including administration functions
such as changing the unit's address, phase offset, etc.


Other software such as 
\codetype{lumos}(1)
or---providing appropriate drivers are installed---popular
third-party programs such as Vixen are more appropriate for
performing (``playing'') sequences of light patterns on
these boards.  By contrast, 
\codetype{lumosctl}
is more suited to setting up and configuring the boards (although some
basic real-time control of channel outputs is possible using
\codetype{lumosctl}).


In the absence of any command-line options to the contrary, the
normal 
operation of
\codetype{lumosctl}
is to make a number of channel output level 
changes as determined by the non-option arguments which are of the
form:
\begin{center}
\Var*{channel}
\end{center}
or
\begin{center}
\Var*{channel}\codetype{@}\Var*{level}\codetype{\textnormal{[},\textnormal{\dots ]}}
\end{center}
or
\begin{center}
\Var*{channel}\codetype{\textnormal{\{}u\textnormal{|}d\textnormal{\}[}:}\Var*{steps}\codetype{\textnormal{[}:}\Var*{time}\codetype{\textnormal{]]}}
\end{center}


In the first case, a channel number by itself means to turn on that 
channel to full brightness.  In the second case, by specifying a level value
(a number from 0 to 255, inclusive), that channel's output is dimmed
to the given level.  Level 255 is the same as turning on to full 
brightness; level 0 is the same as turning it fully off.


In the third case, the dimmer level is ramped up smoothly from its current
value to full brightness 
(``\codetype{u}''),
or down smoothly until fully off
(``\codetype{d}'').
Optionally you may specify the number of dimmer level increments
to increase or decrease at each change (1--128, default is 1); additionally,
you may specify the amount of time to wait between each step, in units
of 1/120 second (1--128, default is 1).  As a convenience, this may be
expressed as a real number of seconds followed by the letter 
``\codetype{s}''.
Thus, the argument
\codetype{13@127}
sets channel 13 to half brightness.  If this were followed by the argument
\codetype{13u}
then channel 13 would be smoothly increased in brightness from there to full
brightness (which is another 128 levels to take it from 127 to 255), 
by incrementing it one level every 1/120th of a second, reaching a full
brightness level 128/120 seconds later (1.0666 seconds).  If the 
argument
\codetype{13d:10:2}
were given, then channel 13 would drop to being fully off, going in steps
of 10 levels at a time, 1/60th of a second between each step.
Finally, an argument
\codetype{10u:5:0.25s}
fades channel 10 up from its current value to full brightness by incrementing
its value by 5 every quarter-second.


Bulk updating of channels is also supported.  If multiple values are listed
for a channel, such as:
\codetype{10@0,0,255,255,127,40,30,20,10}


Then the channel named (10 in this example) is assigned the first value
(0), and the subsequent values are assigned to the immediately following
channels (so channel 11 is set to 0, 12 is set to 255, and so forth).


In addition to the channel-setting argument described above,
an argument of the form
\codetype{x}\Var*{i}
causes stored sequence 
\Var*{i}
to be executed.  Note that this is run in the ``background''---any
subsequent channel-setting arguments will be acted upon 
\emph{while}
the sequence is running.  If a sequence was already running, it is stopped first.
As a special case, 
\codetype{x0}
stops the currently-running sequence but does not start a new one.


A pause in the execution of the arguments may be effected by adding an
argument of the form 
\codetype{p}\Var*{t}\codetype{\textnormal{[}s\textnormal{[}ec\textnormal{[}ond\textnormal{[}s\textnormal{]]]]}}
which makes
\codetype{lumosctl}
pause for
\Var*{t}
seconds before continuing on to the next argument.  The
\Var*{t}
value need not be an integer.


A number of options are provided as described below.  These command
the SSR controller to perform certain administrative functions or 
configuration changes.


When giving multiple types of commands in one invocation of this
program, they will be carried out in the following order:
\begin{enumerate}
% tagged paragraph, width=4
\item
Address Change
% tagged paragraph, width=4
\item
Kill all channels
% tagged paragraph, width=4
\item
Other configuration changes
% tagged paragraph, width=4
\item
Disable privileges
% tagged paragraph, width=4
\item
Channel(s) off/on/dim/etc.
% tagged paragraph, width=4
\item
Shutdown
\end{enumerate}
\section*{OPTIONS}


Each of the following options may be specified by either a long
option (like
``\codetype{----verbose}'')
or a shorter option letter (like
``\codetype{--v}'').
If an option takes a parameter, it may follow the option as
``\codetype{--a12}'',
``\codetype{--a 12}'',
``\codetype{----address 12}'',
or
``\codetype{----address=12}''.


Long option names may be abbreviated to any unambiguous initial substring.
\begin{list}{}{}
% tagged paragraph, width=27
\item[{\codetype{----address=}\Var*{addr}}]\hfill\\
(\codetype{--a}
\Var*{addr})
Specifies the address of the target controller unit.  The
\Var*{addr}
value is an integer from 0 to 15, inclusive.  It defaults to 0.
% tagged paragraph, width=4
\item[{\codetype{----clear--sequences}}]\hfill\\
(\codetype{--S})
Delete all stored sequences from the device's memory.
% tagged paragraph, width=4
\item[{\codetype{----disable--sensor=}\Var*{s}}]\hfill\\
(\codetype{--D}
\Var*{s})
Disable inputs from the sensor(s) specified as the
\Var*{s}
parameter (which are given as a set of one or more letters,
e.g., 
\codetype{----disable--sensor=ab}).
The Lumos board will act as though those sensors were 
inactive regardless of their actual inputs.
The special character
``\codetype{*}''
appearing in 
\Var*{s}
means to disable all sensors.
% tagged paragraph, width=4
\item[{\codetype{----drop--privileged--mode}}]\hfill\\
(\codetype{--d})
If the Lumos device is in privileged command mode (for configuraiton of
the device), this will cancel that mode.  Further privileged commands will
not be recognized on that device.
% tagged paragraph, width=4
\item[{\codetype{----dump--configuration=}\Var*{file}}]\hfill\\
(\codetype{--C}
\Var*{file})
Dump the device configuration into the named
\Var*{file}.
See below for a description of the configuration file format.
% tagged paragraph, width=4
\item[{\codetype{----enable--sensor=}\Var*{s}}]\hfill\\
(\codetype{--E}
\Var*{s})
Enable inputs from the sensor(s) specified as the 
\Var*{s}
parameter.  See 
\codetype{----disable--sensor}.
% tagged paragraph, width=4
\item[{\codetype{----help}}]\hfill\\
(\codetype{--h})
Prints a summary of these options and exits.
% tagged paragraph, width=4
\item[{\codetype{----kill--all}}]\hfill\\
(\codetype{--k})
Turn off all output channels at once.
% tagged paragraph, width=4
\item[{\codetype{----load--configuration=}\Var*{file}}]\hfill\\
(\codetype{--c}
\Var*{file})
Load the device configuration from the named
\Var*{file}
and program that into the device.
% tagged paragraph, width=4
\item[{\codetype{----load--hex--sequence=}\Var*{file}}]\hfill\\
(\codetype{--H}
\Var*{file})
Load one or more pre-compiled sequences from the specified hex 
\Var*{file}.
This is expected to be the output from the 
\codetype{lumosasm}(1)
command.
% tagged paragraph, width=4
\item[{\codetype{----load--sequence=}\Var*{file}}]\hfill\\
(\codetype{--s}
\Var*{file})
Load one or more sequences from the specified source
\Var*{file}
(see below for sequence source code syntax) and program them
into the device.  If another sequence already exists with the
same number, it replaces the old one; however, beware that the
controller device does not optimize memory storage, so eventually
stored sequences may become fragmented, resulting in running out of
storage space for them.  To avoid this, it is best to clear all
sequences using the
\codetype{----clear--sequences}
option, then load all the sequences you want on the device at once.
% tagged paragraph, width=4
\item[{\codetype{----port=}\Var*{port}}]\hfill\\
(\codetype{--p}
\Var*{port})
Specify the serial port to use when communicating with the controller
unit.  This may be a simple integer value (0 for the first serial port
on the system, 1 for the next one, etc.) or the actual device name
on your system (such as 
``\codetype{COM1}''
or
``\codetype{/dev/ttys0}'').
% tagged paragraph, width=4
\item[{\codetype{----probe}}]\hfill\\
(\codetype{--P})
Search for, and report on, all Lumos controllers attached to the serial network.
If the
\codetype{----report}
option is also specified, this provides that level of output for every attached
device; otherwise, it only lists device models and addresses.
% tagged paragraph, width=4
\item[{\codetype{----report}}]\hfill\\
(\codetype{--R})
Report on the current device status to standard output in human-readable form.
% tagged paragraph, width=4
\item[{\codetype{----sensor=}\Var*{s}\codetype{\textnormal{[}orw+\textnormal{]}:}\Var*{init}\codetype{:}\Var*{seq}\codetype{:}\Var*{term}}]\hfill\\
(\codetype{--t}
\Var*{s}\codetype{\textnormal{[}orw+\textnormal{]}:}\Var*{init}\codetype{:}\Var*{seq}\codetype{:}\Var*{term}\codetype{\textnormal{)}}
Define an action to be taken when a sensor is triggered.  When the sensor
is activated, the sequence
\Var*{init}
is run, followed by the sequence
\Var*{seq}
and then finally the sequence
\Var*{term}
when the sensor event is over.  The sensor assigned this action is given
as the parameter
\Var*{s}
and is one of the letters 
\codetype{A},
\codetype{B},
\codetype{C},
or
\codetype{D}.
This may be followed by the following option letters as needed:
\begin{itemize}
% tagged paragraph, width=4
\item[{\codetype{o}}]
Trigger once: play sequence
\Var*{seq}
only one time.  The action will not be taken again until the sensor
input transitions to inactive and then asserts itself as active again.
This is the default action.
% tagged paragraph, width=4
\item[{\codetype{r}}]
Repeat mode: play sequence
\Var*{seq}
indefinitely until explicitly told to stop (by an overt stop command
such as an
\codetype{x0}
argument, or another sequence being triggered manually or by sensor
action).
% tagged paragraph, width=4
\item[{\codetype{w}}]
Trigger while active: play sequence
\Var*{seq}
repeatedly as long as the sensor remains active.  When the sensor input
transitions to inactive again, terminate the action.
% tagged paragraph, width=4
\item[{\codetype{+}}]
The sensor is to be considered ``active'' when at a logic
high output (active-high mode).  Normally, sensors are active-low
(active when the input is at ground).
\end{itemize}


If 0 is specified for any of the sequence numbers, that means no
sequence is called for that part of the trigger action.
% tagged paragraph, width=4
\item[{\codetype{----set--address=}\Var*{addr}}]\hfill\\
(\codetype{--A}
\Var*{addr})
Change the device address to
\Var*{addr}.
This must be an integer in the range 0--15.
% tagged paragraph, width=4
\item[{\codetype{----set--baud--rate=}\Var*{speed}}]\hfill\\
(\codetype{--B}
\Var*{rate})
Set a new baud rate for the device to start using from now on.
% tagged paragraph, width=4
\item[{\codetype{----set--phase=}\Var*{offset}}]\hfill\\
(\codetype{--P}
\Var*{offset})
Set the phase 
\emph{offset}
in the device to the specified value.  This must be an integer in the range 0--511.
\emph{This is an advanced setting which affects the ability of the AC relay boards to}
\emph{function properly. Do not change this setting unless you know exactly what you are doing.}
% tagged paragraph, width=4
\item[{\codetype{----sleep}}]\hfill\\
(\codetype{--z})
Tell the unit to go to sleep (this instructs the board to turn off a
power supply which it is controlling, if any, but has no other effect).
% tagged paragraph, width=4
\item[{\codetype{----shutdown}}]\hfill\\
(\codetype{--X})
Command the unit to shut down completely.  It will be unresponsive until
power cycled or the reset button is pressed to reboot the controller.
% tagged paragraph, width=4
\item[{\codetype{----speed=}\Var*{rate}}]\hfill\\
(\codetype{--b}
\Var*{rate})
Set the serial port to the given
baud
\Var*{rate}.
[Default is 19200 baud.]
% tagged paragraph, width=4
\item[{\codetype{----wake}}]\hfill\\
(\codetype{--w})
Tell the unit to start the attached power supply from sleep mode.
command is given at a future time.
% tagged paragraph, width=4
\item[{\codetype{----verbose}}]\hfill\\
(\codetype{--v})
Output messages to the standard output.  Additional 
\codetype{----verbose}
options increases verbosity.  High levels of verbosity include a dump
of every bit sent or received on the serial network.
\end{list}
\section*{CONFIGURATION FILE FORMAT}


The files read and written by the 
\codetype{----dump--configuration}
and
\codetype{----load--configuration}
options use a fairly standard configuration file format similar to the
``ini'' files used by early versions of Microsoft Windows and other 
systems.  For full details of this format see
\codetype{http://docs.python.org/library/configparser.html},
but the highlights include:
\begin{enumerate}
% tagged paragraph, width=3
\item
One data value per line (long lines may be continued by indentation ala RFC 822 headers).
% tagged paragraph, width=4
\item
Each line consists of the name of a data value, either an equals sign or a colon,
and the value itself.
% tagged paragraph, width=4
\item
A syntax 
\codetype{\%(}\Var*{name}\codetype{)s}
can be used to substitute values into other values.  Literal percent signs in values
are simply doubled 
(``\codetype{\%\%}'').
\end{enumerate}


All configration data are contained in a stanza called
``\codetype{[lumos\_device\_settings]}''.
The values are:
\begin{list}{}{}
% tagged paragraph, width=4
\item[{\codetype{baud=}\Var*{n}}]\hfill\\
The configured serial I/O speed of the device.  Supported values include 300, 600, 1200, 2400,
4800, 9600, 19200, 38400, 57600, 115200, and 250000.  Speeds slower than 9600 baud
are not recommended. [Default is 19200.]
% tagged paragraph, width=4
\item[{\codetype{dmxchannel=}\Var*{n}}]\hfill\\
If this field exists, the Lumos board is to run in DMX512 mode, with its channel \#0 appearing
at DMX512 slot 
\#\Var*{n},
where
\Var*{n}
is an integer in the range 1--512.
If this field is not present, the Lumos board will not be configured to recognize DMX512 packets
at all.
% tagged paragraph, width=4
\item[{\codetype{phase=}\Var*{offset}}]\hfill\\
The AC waveform phase offset for the unit.  This should only be changed if needed due to some
anomaly with the zero-crossing detector which throws off the unit's timing.  This is an integer
in the range 0--511.  [Default is 2.]
% tagged paragraph, width=4
\item[{\codetype{sensors=}\Var*{list}}]\hfill\\
The value is a list of single letters in the range A--D.  Each letter appearing in this
list indicates that the corresponding sensor input should be enabled in the hardware.
You must ensure that the hardware is really configured that way.  
\end{list}
\subsection*{Sensor Configuration}


For each sensor listed in the 
\codetype{sensors}
field, a corresponding stanza called 
``\codetype{[lumos\_device\_sensor\_\Var*{x}]}''
appears, where 
\Var*{x}
is the name of the sensor 
(``\codetype{A}'',
``\codetype{B}'',
``\codetype{C}'',
or
``\codetype{D}''),
with the following fields:
\begin{list}{}{}
% tagged paragraph, width=4
\item[{\codetype{enabled=}\Var*{bool}}]\hfill\\
If 
``\codetype{yes}'',
the sensor input is set to be monitored.  If
``\codetype{no}'',
it is ignored.  [Default is 
``\codetype{yes}''.]
% tagged paragraph, width=4
\item[{\codetype{mode=}\{\codetype{once}|\codetype{repeat}|\codetype{while}\}}]\hfill\\
Define the operating mode of the sensor trigger: play once per trigger,
repeat forever until another trigger (or explicit command to stop), or
play as long as sensor remains active. [Default is 
\codetype{once}.]
% tagged paragraph, width=4
\item[{\codetype{setup=}\Var*{id}}]\hfill\\
Sequence
\Var*{id}
number to be played initially when the sensor becomes active
% tagged paragraph, width=4
\item[{\codetype{sequence=}\Var*{id}}]\hfill\\
Sequence
\Var*{id}
number to be played as the main (possibly repeated) aciton for the sensor.
% tagged paragraph, width=4
\item[{\codetype{terminate=}\Var*{id}}]\hfill\\
Sequence
\Var*{id}
number to be played when the action stops. Note that the
main sequence might not have played to completion.
\end{list}
\section*{SEQUENCE SOURCE SYNTAX}


Each source file given to 
\codetype{----load--sequence}
contains one or more sequence definitions as described here.  The formal syntax
definition for the sequence language is:
\begin{SourceCode}
<sequence> ::= SEQUENCE <id> ['(' <arg-list> ')'] <block>
<statement-list> ::= <statement> <newline> 
    | <statement-list> <statement> <newline>
<statement> ::= BLACKOUT
    | CHANNEL <chan> <state>
    | RAMP <direction> <chan> [BY <steps> [PER <time>]]
    | CALL <id> ['(' <value-list> ')']
    | EXECUTE <id> ['(' <value-list> ')']
    | WAIT <time>
    | FOR <symbol> '=' <value> TO <value> [BY <value>] <block>
    | IF <condition> <block>
    | UNLESS <condition> <block>
    | REPEAT [<value> [TIME[S]]] [AS <symbol>] <block>
    | BREAK <symbol>
    | CONTINUE <symbol>
    | <symbol> '=' <value>
    | SLEEP
    | WAKE
    | SUSPEND [WITH UPDATE]
    | RESUME [WITH FADE]
    | EXIT
<block> ::= ':' <newline> <statement-list> END
<id> ::= 0 | 1 | ... | 127
<chan> ::= <value> (allowed range 0...63)
<dimmer-value> ::= 0 | 1 | ... | 255
<percentage> ::= 0 | 1 | ... | 100
<state> ::= ON | OFF | <dimmer-value> | <percentage> '%'
<direction> ::= UP | DOWN
<steps> ::= 1 | 2 | ... | 128
<time> :: = <cycles> | <real> SEC[ONDS]
<cycles> ::= 1 | 2 | ... | 128
<real> ::= <digits> ['.' [<digits>]] | [<digits>] '.' <digits>
<condition> ::= SUSPENDED
    | SENSOR <sens-list>
    | <value> <comparison> <value>
<comparison> ::= '<' | '>' | '==' | '!=' | '<=', '>='
<arg-list> ::= <symbol> | <arg-list> ',' <symbol>
<value-list> ::= <value> | <value-list> ',' <value>
<value> ::= <int> 
    | <symbol>
    | <value> '*' <value>
    | <value> '+' <value>
    | <value> '-' <value>
    | <value> '/' <value>
    | '(' <value> ')'
<int> ::= ['-' | '+'] <digits>
<symbol> ::= <letter> | <symbol> <digit> | <symbol> <letter>
<digits> ::= <digit> | <digits> <digit>
<digit> ::= '0' | '1' | ... | '9'
<letter> ::= '_' | 'A' | 'B' | ... | 'Z'
<sens-list> ::= <sens> | <sens-list> OR <sens>
<sens> ::= 'A' | 'B' | 'C' | 'D'
<newline> ::= '\n'
\end{SourceCode}


While the language keywords are shown here in upper-case letters,
in fact all symbols are interpreted irrespective of case throughout
the sequence source file.


\subsection*{Expressions}
Simple math expressions are supported, including addition, subtraction,
multiplication, and division, but the values supported by the hardware
runtime system are exclusively 8 bit unsigned integers.  Division is not
supported at all in the hardware.  Math involving compile-time values
is not so limited, but if the expression evaluation remaining to be done
at runtime it too complex, the compiler will issue a fatal error.
\subsection*{Flow Control Constructs}


Each of these constructs operates on a 
\Var*{block}
of code.  A block is a sequence of one or more lines of code (each terminated
by a newline), ending with an 
\codetype{END}
statement.


\begin{list}{}{}
% tagged paragraph, width=4
\item[{\codetype{IF SENSOR }\Var*{s1}\codetype{ \textnormal{[}OR }\Var*{s2}\codetype{\textnormal{] [\dots ]}: \textnormal{\dots } END}}]\hfill\\
Execute the statements in the block if any of the sensors are currently active.
% tagged paragraph, width=4
\item[{\codetype{IF SUSPENDED: }\dots \codetype{ END}}]\hfill\\
Execute the block if the unit is currently in a suspended state.
% tagged paragraph, width=4
\item[{\codetype{IF }\Var*{condition}\codetype{: \textnormal{\dots } END}}]\hfill\\
Execute the block if the condition holds. This is a simple comparison between
two values, where the comparison may be equals, not equals, less than, greater than,
less than or equal, or greater than or equal as represented by the operators
\codetype{==},
\codetype{!=},
\codetype{<},
\codetype{>},
\codetype{<=},
and
\codetype{>=},
respectively.
% tagged paragraph, width=4
\item[{\codetype{UNLESS }\dots}]\hfill\\
All of the forms of the
\codetype{IF}
construct may be used with the word 
\codetype{UNLESS}
substituted for
\codetype{IF}.
In this case, the sense of the conditional test is reversed.
% tagged paragraph, width=4
\item[{\codetype{REPEAT \textnormal{[}}\Var*{n}\codetype{ \textnormal{[}TIME\textnormal{[}S\textnormal{]]} \textnormal{[}AS }\Var*{var}\codetype{\textnormal{]}: \textnormal{\dots } END}}]\hfill\\
Repeat the statements in the block until forced to quit via 
\codetype{EXIT}
or
\codetype{BREAK}.
If a repeat count
\Var*{n}
is given, then at most only repeat the statements that many times.
The 
\codetype{AS}
clause allows the current loop counter (which begins at 0 and counts up to, 
but never reaches,
\Var*{n})
to be visible within the block as the variable
\Var*{var}.
% tagged paragraph, width=4
\item[{\codetype{FOR }\Var*{var}\codetype{=}\Var*{start}\codetype{ TO }\Var*{end}\codetype{ \textnormal{[}BY }\Var*{step}\codetype{\textnormal{]}: \textnormal{\dots } END}}]\hfill\\
Repeat the statements in the block once for each value of 
\Var*{var}
from
\Var*{start}
to 
\Var*{end},
inclusive.
If
\Var*{step}
is given, 
\Var*{var}
is incremented by that amount between each iteration.  The default
\Var*{step}
is 1.
The behavior of this loop may be modified by
\codetype{BREAK}
or
\codetype{CONTINUE}
statements just like the
\codetype{REPEAT}
loops are.
% tagged paragraph, width=4
\item[{\codetype{BREAK \textnormal{[}}\Var*{var}\codetype{\textnormal{]}}}]\hfill\\
Terminate the innermost loop immediately.  If an outer loop is to be terminated,
then specify its index 
\Var*{var}
to identify which loop is the target of this statement.
% tagged paragraph, width=4
\item[{\codetype{CONTINUE \textnormal{[}}\Var*{var}\codetype{\textnormal{]}}}]\hfill\\
Begin the next iteration of the innermost loop immediately.  
If an outer loop is to be continued,
then specify its index 
\Var*{var}
to identify which loop is the target of this statement.
% tagged paragraph, width=4
\item[{\codetype{SEQUENCE }\Var*{id}\codetype{\textnormal{[}(}\Var*{var}\codetype{, \textnormal{\dots })\textnormal{]}: \textnormal{\dots } END}}]\hfill\\
Define a stored sequence consisting of the statements to the 
\codetype{END}.
The 
\Var*{id}
may be a number from 0 to 127, with a few caveats:
\begin{description}
% tagged paragraph, width=4
\item[{\#0:}]
Sequence \#0 may 
\emph{never}
be explicitly invoked by anything.  This sequence is invoked automatically
during device boot to initialize the unit.
% tagged paragraph, width=4
\item[{\#1--63:}]
Sequences in this range are stored in EEPROM and will survive a power failure
or device reboot.  Note that there probably won't be enough available memory
to actually store 63 sequences.
% tagged paragraph, width=4
\item[{\#64--127:}]
Sequences in this range are stored in RAM and will 
\emph{not}
survive a power failure
or device reboot.  Note that there probably won't be enough available memory
to actually store 64 sequences in RAM. 
\end{description}
\end{list}


If input parameters are expected, they are given symbolic names inside
parentheses between the sequence
\Var*{id}
and the colon.  A maximum of four parameters are allowed.
\subsection*{Commands}


Each statement is described briefly here.  For more details, see the
Lumos board user manual.
\begin{list}{}{}
% tagged paragraph, width=4
\item[{\codetype{BLACKOUT}}]\hfill\\
Immediately turn off all channel outputs.
% tagged paragraph, width=4
\item[{\codetype{CALL }\Var*{id}\codetype{\textnormal{[}(}\Var*{value}\codetype{,\textnormal{ \dots })\textnormal{]}}}]\hfill\\
Suspend the execution of the current sequence and begin executing
sequence
\#\Var*{id}.
When that sequence has completed, execution of the current sequence
will resume where it left off.  Up to four parameters may be passed
to the called sequence by placing them inside parentheses after the
sequence 
\Var*{id}.
The 
\Var*{id}
cannot be 0.
% tagged paragraph, width=4
\item[{\codetype{CHANNEL }\Var*{c}\codetype{ ON|OFF|}\Var*{value}\codetype{\textnormal{[}\%\textnormal{]}}}]\hfill\\
Change the output level of channel
\Var*{c}
to fully on, fully off, or to a specific value.  If the value
is followed by a percent sign 
(``\codetype{\%}''),
then the value is assumed to be a percentage in the range 0--100.  
% tagged paragraph, width=4
\item[{\codetype{EXECUTE }\Var*{id}\codetype{\textnormal{[}(}\Var*{value}\codetype{,\textnormal{ \dots })\textnormal{]}}}]\hfill\\
Abandon the current sequence and begin executing the specified sequence
instead.
When that sequence has completed, execution of the current sequence
will not resume.  Up to four parameters may be passed
to the new sequence by placing them inside parentheses after the
sequence 
\Var*{id}.
Note that the 
\Var*{id}
cannot be 0.
% tagged paragraph, width=4
\item[{\codetype{EXIT}}]\hfill\\
Terminate execution of the current sequence.
% tagged paragraph, width=4
\item[{\codetype{RAMP UP|DOWN }\Var*{c}\codetype{ \textnormal{[}BY }\Var*{steps}\codetype{ \textnormal{[}PER }\Var*{time}\codetype{ \textnormal{[}SEC\textnormal{[}ONDS\textnormal{]]]}}}]\hfill\\
Start fading channel
\Var*{c}
up or down from its current output level to the maximum or minimum value.  This is done
by adding or subtracting the value of
\Var*{steps}
(which is in the range 1--128) each time, pausing for
\Var*{time}
between each value change.  If the
\Var*{time}
value is followed by 
\codetype{SEC}
or
\codetype{SECONDS},
it is assumed to be the number of seconds between level changes (a real number
in the range 0.0083--1.0667). 
Otherwise, it is assumed to be an integer in the range 1--128 specifying the number
of 1/120ths of a second for the pause.  
% tagged paragraph, width=4
\item[{\codetype{RESUME }[\codetype{WITH FADE}]}]\hfill\\
Resume from
\codetype{SUSPEND}
mode.  If the
\codetype{WITH FADE}
option is given, then all channels are faded down to fully off, and then up to
their last-known state is supposed to be.
% tagged paragraph, width=4
\item[{\codetype{SLEEP}}]\hfill\\
Enter sleep state.  The load power supply is commanded to shut down (if such
a supply is connected that way to the controller).  Note that the unit may still
wake on its own as needed, possibly even immediately.
% tagged paragraph, width=4
\item[{\codetype{SUSPEND }[\codetype{WITH UPDATE}]}]\hfill\\
Suspend input of commands from outside.  From this point forward, only saved
sequences will affect channel outputs.  If the
\codetype{WITH UPDATE}
option is given, then external commands are still accepted but have no effect
on the outputs until after a
\codetype{RESUME}
is executed.  Note that a unit may still remain in suspended state after the
sequence putting it in that state exits.  It only ends upon
\codetype{RESUME}.
% tagged paragraph, width=4
\item[{\codetype{WAIT \Var*{t} }[\codetype{SEC}[\codetype{ONDS}]]}]\hfill\\
Pause execution of the script for the designated time.  The time
\Var*{t}
is interpreted identically to that described above for the
\codetype{RAMP}
command.
% tagged paragraph, width=4
\item[{\codetype{WAKE}}]\hfill\\
Wake up the unit from
\codetype{SLEEP}
mode.  The power supply is commanded to start (if a suitable one is appropriately
configured).
\end{list}
\section*{AUTHOR}


Software Alchemy / support@alchemy.com
\section*{COMPATIBILITY}


This version of 
\codetype{lumosctl}
is compatible with the following boards:
\begin{itemize}
% tagged paragraph, width=2
\item
Lumos 48-channel controller version 3.1 or 3.2
\emph{providing it has been upgraded or installed with ROM firmware version 3.0 or later}
(boards with ID markings beginning with ``48CTL-3-1'' or ``LUMOS-48CTL-3.2'').  (Whether this controller is
driving AC or DC boards is irrelevant.)
% tagged paragraph, width=4
\item
Lumos 24-channel DC controller version 1.0 (boards with ID markings beginning with
``LUMOS-24SSR-DC-1.0'').
\end{itemize}
\section*{HISTORY}



This program first appeared under the name
\codetype{48ssrctl}
and was used only for the Lumos 48-channel AC controller boards, employing the
older firmware (ROM versions less than
3.\emph{x}).


This document describes version 2.0 of this utility, which is the first to carry
this name and to include the expanded features for firmware version 3.0.
\section*{SEE ALSO}
\codetype{lumosasm}(1).
\section*{LIMITATIONS}


This program does not send DMX512 commands to the device(s), only Lumos native
commands.
\section*{BUGS}


The sequence language is constrained by the limits of the hardware (such as 8-bit
unsigned integer values and limited arithmetic expression evaluation), 
by the need to 
be compiled to fit in a very small memory space.  As such, the optimization
toward certain use cases and against others may seem odd at first, but it serves that
purpose.


Submit any other issues found to 
\codetype{support@alchemy.com}.
