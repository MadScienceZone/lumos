% When sizing images, a full-width image is about 126 mm, half-size
% are about 63 mm, full height about 170 mm.
%
% Using the circuit boards for hardcovers changes the page size.  The
% board is 267x178mm, so leaving 5mm at the top and bottom (and assuming
% the paper will extend 5mm into the spine past the hinge point), that
% make the new page size 257x178.
%
% LiveJournal book typesetting macros
% version 1.0 
% Steve Willoughby <steve@alchemy.com> 12/2009
%
% Provides the following (see below for more details)
% COMMANDS:
%  \acronym[*]{TEXT}                        Typeset TEXT as an acronym (\mc)  + index automatically unless *
%  \asidenote{text}                         Stage-whispers to reader
%  \footnoteB{text}                         2nd tier of footnotes
%  \ix{text}                                Typeset <text> plus do \index{<text>}
%  \LLimg[opts]{filename}                   Include image (wraps \epsfig)  opts: width= etc
%  \LLquote[*!][subtext]{[/@]author}{text}  Epigraph (*=wide, !=narrow)
%  \LLquotex[subtext]{[/@]author}{text}{w}  Eipgraph (random paragraph width)
%                                             {/author}  suppress indexing
%                                             {@author}  suppress "Last, First"
%                                             {x!y}      typeset x, index as y
%                                             {x!}       typeset x, do not index
%  \LLqindex*                               Suppress author indexing for next \LJquote
%  \LLqindex{entry}                         Index next \LJquote as <entry>, typeset author as-is in \LJquote arg
%  \LLstart{versal}{word}{leader}           Initial paragraph of entry
%  \LLstart*                                Begin entry text w/o initial par.
%
%  \LLversal{versal}{word+leader}           "Initial" paragraph after \LJstart
%  \marginComment{text}                     Simple comment, but in margins
%  \mc{TEXT}                                Set TEXT in medium caps
%  \Qauthor[subtext]{author}                Give credit to author of quotation.
%  \starsep[char]                           Decorative * * * * separator
%  \strong{text}                            Typeset with strong emphasis
%  \TM                                      Trademark symbol
%  \URL{url}                                Set url name (tt font)
%  \URLref[*][text]{url}                    text refs URL in footnote (*=B)
%  \Var[*]{name}                            Variable (*=bracketed)
%  \work[*]{title}                          Cited work name (Italics)  + index automatically unless *
%
% ENVIRONMENTS:
%  SourceCode                               Print block of source code in a box
%  Coding                                   Use |verbatim| and \Var in env
%                                            also \Var* puts <> around variable name
%
% LABELS:
%\flushbottom
%\raggedbottom
%\setstocksize{8.5in}{7in}
%\settrimmedsize{8.5in}{7in}{*}
%\checkandfixthelayout
\usepackage{ifpdf}
\usepackage{comment}
\usepackage{multirow}
\usepackage{ifthen}
\usepackage{relsize}
\usepackage{rotating}
\usepackage{indent}
\usepackage{bigfoot}
\usepackage[normalem]{ulem}
\usepackage{suffix}
\usepackage{wrapfig}
\usepackage{epsfig}
\usepackage{textcomp}
\usepackage{lettrine}
\usepackage[endianness=big]{bytefield}
\usepackage{bbding}
\usepackage{substr}
%\usepackage{xstring}
\usepackage[T1]{fontenc}
%\usepackage{palatino}
%\usepackage{bera}
\usepackage{tgschola}
%\usepackage{tgpagella}
%\usepackage{fouriernc}
%\usepackage{CJK}
%\usepackage{pinyin}
%\usepackage{aurical}
%\usepackage{emerald}
\usepackage{colortbl}
\usepackage{longtable}
\usepackage{color}
\usepackage{calc}
\usepackage{graphicx}
\usepackage{soul}
\usepackage{xfrac}
\usepackage{wallpaper}
%\usepackage{tikz}
\usepackage{circuitikz}	
%\makeatletter
%\let\pinyinXi\Xi
%\let\pinyinPi\Pi
%\let\Xi\@undefined
%\let\Pi\@undefined
%\makeatother
%\usepackage{fourier}
\makeindex
%\makeindex[tags]
%\makeindex[comments]
%\makeindex[quotes]
\newcommand\ix[1]{#1\index{#1}}
%\makeglossary
\showindexmarks
%\nouppercaseheads
\hideindexmarks
%
% Page Layout Parameters
%
\setstocksize{11in}{8.5in}        % printing on letter stock  8.5x11
%\settrimmedsize{234mm}{180mm}{*}  % trimmed paper dimensions ~7.5x9.5
%
% old sizes
%
%\settrimmedsize{234mm}{184mm}{*}  % trimmed paper dimensions ~7.5x9.5
%\settrims{23mm}{35mm}             % offset into stock of trimmed paper
%\settypeblocksize{182mm}{32pc}{*} % dimensions of type block
%\setlrmargins{*}{27mm}{*}         % margins 27mm from outer edge
%\setulmargins{*}{27mm}{*}         % margins 27mm from bottom edge
%\setheaderspaces{*}{12mm}{*}      % 12mm between header and text baselines
%\setmarginnotes{5mm}{15mm}{2mm}   % margin notes 15mm wide, 5mm from text, 2mm from each other
%
% new sizes
%
%\settrimmedsize{257mm}{180.5mm}{*}% trimmed paper dimensions 
%\settrims{12mm}{37mm}             % offset into stock of trimmed paper
%\settypeblocksize{205mm}{31pc}{*} % dimensions of type block
%\setlrmargins{*}{27mm}{*}         % margins 27mm from outer edge
%\setulmargins{*}{27mm}{*}         % margins 27mm from bottom edge
%\setheaderspaces{*}{12mm}{*}      % 12mm between header and text baselines
%\setmarginnotes{5mm}{15mm}{2mm}   % margin notes 15mm wide, 5mm from text, 2mm from each other
\fixpdflayout
%
\trimFrame
\checkandfixthelayout

\newcommand\defaultchaptercolor{%
  \definecolor{chapterboxcolor}{rgb}{.647,.129,.149}%
}
\defaultchaptercolor
\makeatletter
\newlength\dlf@normtxtw
\setlength\dlf@normtxtw{\textwidth}
\def\myhelvetfont{\def\fdefault{mdput}}
\newsavebox{\feline@chapter}
\newcommand\feline@chapter@marker[1][4cm]{%
\sbox\feline@chapter{%
\resizebox{!}{#1}{\fboxsep=1pt%
%\colorbox{chapterboxcolor}{\color{white}\bfseries\sffamily\thechapter}%
\colorbox{chapterboxcolor}{\color{white}\bfseries\thechapter}%
}}%
\rotatebox{90}{%
\resizebox{%
\heightof{\usebox{\feline@chapter}}+\depthof{\usebox{\feline@chapter}}}%
{!}{\scshape\so\@chapapp}}\quad%
\raisebox{\depthof{\usebox{\feline@chapter}}}{\usebox{\feline@chapter}}%
}
\newcommand\feline@chm[1][4cm]{%
\sbox\feline@chapter{\feline@chapter@marker[#1]}%
\makebox[0pt][l]{% aka \rlap
\makebox[1cm][r]{\usebox\feline@chapter}%
}}
\makechapterstyle{daleif1}{
\renewcommand\chapnamefont{\normalfont\Large\scshape\raggedleft\so}
\renewcommand\chaptitlefont{\normalfont\huge\bfseries\scshape\color{chapterboxcolor}}
\renewcommand\chapternamenum{}
\renewcommand\printchaptername{}
\renewcommand\printchapternum{\null\hfill\feline@chm[2.5cm]\par}
\renewcommand\afterchapternum{\par\vskip\midchapskip}
\renewcommand\printchaptertitle[1]{\chaptitlefont\raggedleft ##1\par}
}
\makeatother
\chapterstyle{daleif1}


%\chapterstyle{veelo}
%\stockletter      % 8.5 x 11  in stock paper size
%\pagestatement    % 8.5 x 5.5 in page size
%\settrimmedsize{9in}{6in}{*}
%\settrims{1in}{1.25in}
%\setlrmarginsandblock{.5in}{.5in}{*}
%\setulmarginsandblock{1in}{1in}{*}
%\trimFrame
%\quarkmarks
%
% Footnote Setup
%
\DeclareNewFootnote{default}
\DeclareNewFootnote{B}[fnsymbol]
\MakeSortedPerPage{footnoteB}

%
% Fix table of contents spacing
% @pnumwidth [1.55em] width of box holding page numbers in TOC
% @tocrmarg  [2.55em] title text indentation from right margin
% @dotsep    [4.5]    math units of space between TOC leader dots.
%
\makeatletter
\renewcommand{\@pnumwidth}{2.5em}
\makeatother
%
%
%
% LLquoteblock environment:
%   Sets text flush-right in Italic font.  Use \Qauthor to give credit
%   after each quote in the block.
%

\setlength{\epigraphrule}{0pt}
%
% \Qauthor[subtext]{author}
%    Give credit to the author of a quotation.  Optional subtext set
%    on the following line in smaller type.
%
\newcommand\Qauthor[2][]{%
  \\%
  \textnormal{---#2%
    \ifthenelse{\equal{#1}{}}{}{%
      \\%
      {\footnotesize #1}%
    }%
  }%
}
%
% Indexing
% \ix{<text>} typesets <text> and indexes it too (identical in both
%             cases--to get something different, use \index explicitly).
%
% \LLquote[subtext]{author}{quote text}
%    Epigraph (a single quote).  The optional subtext is set smaller
%    in upright shape under the author.  Authors are given an em-dash
%    in caps and small caps fully flush right.  The quote text itself
%    is in Italics in a 40% width paragraph which is itself against the
%    right margin.
%
%    The author's name is indexed.  The following special syntax is
%    recognized for the {author} parameter:
%XXX    <text>      - index <text> after trying to change* to Lastname, First
%XXX    @<text>     - index <text> as-is*
%XXX    /<text>     - typeset <text> but do NOT index (use \index manually)
%
%XXX    <text>!<i>  - typeset as <text>, index as <i> using all the usual
%                     indexing controls in <i>.
%XXX    <text>!     - typeset <text> but do NOT index.
%
%    (*These functions are handled by external scripts which prepare the
%    indexes)
%
% \LLquote*[subtext]{author}{quote text}
%    Alternative for wider quotations (uses 60% width).
%
% \LLquote![subtext]{author}{quote text}
%    Alternative for narrow quotations (uses 25% width).
%
% \LLquotex[subtext]{author}{quote text}{width}
%    Extended and generalized form allowing you to specify your own
%    paragraph width.
%
           \newcommand\LLquote[3][]{\LLquotex[#1]{#2}{#3}{0.400\textwidth}}
\WithSuffix\newcommand\LLquote*[3][]{\LLquotex[#1]{#2}{#3}{0.60\textwidth}}
\WithSuffix\newcommand\LLquote![3][]{\LLquotex[#1]{#2}{#3}{0.25\textwidth}}
%
\newcommand\SuppressLjQuoteIndexing{norm}
\newcommand\LLqindex[1]{%
  \index[quotes]{#1}%
  \renewcommand\SuppressLjQuoteIndexing{nil}%
}
\WithSuffix\newcommand\LLqindex*{\renewcommand\SuppressLjQuoteIndexing{nil}}

\newcommand\LLquotex[4][]{%
  \setlength{\epigraphwidth}{#4}%
  \ifthenelse{\equal{\SuppressLjQuoteIndexing}{norm}}{%
    \index[quotes]{#2}%
    \IfCharInString{@}{#2}{%
      \IfBeforeSubStringEmpty{@}{#2}{%
        \def\LLqAuthor{\BehindSubString{@}{#2}}%
      }{%
        \def\LLqAuthor{#2}%
      }%
    }{%
      \def\LLqAuthor{#2}%
    }%
  }{%
    \renewcommand\SuppressLjQuoteIndexing{norm}%
    \def\LLqAuthor{#2}%
  }%
  \ifthenelse{\equal{#1}{}}{%
    \epigraph{{\em #3}}{{---\LLqAuthor}}%
  }{%
    \epigraph{{\em #3}}{{---\LLqAuthor}\\
{\footnotesize #1}}%
  }%
}

% \LLqindex*    Do not index author name at all for next quote
% \LLqindex{@i} Use <i> as index entry for next \LLquote
% \LLqindex{i}  Use <i> as index entry for next \LLquote, changing to Last, First format
% \LLquote[work]{author}{text}  Set quote and index author in Last, First format
% \LLquote[work]{@author}{text} Set quote and index author as-is

%\maxsecnumdepth{chapter}

% SourceCode environment:
%   Typesets source code in a verbatim box.
%
\newenvironment{SourceCode}{\boxedverbatim}{\endboxedverbatim}
\newcommand\codetype[1]{{\ttfamily #1}}
%
% Coding environment:
%   Code verbatim inside |bars|
%   Variables in \Var[*]{...} command  * form uses <> brackets around name
%   
\newenvironment{Coding}{\MakeShortVerb{\|}}{\DeleteShortVerb{\|}}
\newcommand\Var[1]{{\itshape #1\/}}
\WithSuffix\newcommand\Var*[1]{$\langle${\itshape #1\/}$\rangle$}
\newcommand\marginComment[1]{%
  \marginpar{%
    \ECFAugie%
    \noindent #1%
  }%
}
% \LLstart{versal}{first word}{emphasized text}
%   Begin an entry with an initial dropped versal, plus some
%   style applied to the rest of the first word and some 
%   starting text.
%
\newcommand\LLstart[3]{\LLversal{#1}{\lowercase{#2 #3}}}

%\noindent{\tiny [ \box\ljAttrBox\ ]}\par}
%
% \LLversal{versal}{first word(s)}
%   Begin a chapter paragraph.  This is just like \LLstart but without the
%   meta-post information being printed.
%   (In fact, \LLstart calls \LLversal to do the initial text.)
%
\newcommand\LLversal[2]{%
 \ifthenelse{\equal{#1}{A}}{%
  \lettrine[lines=3,findent=-1.1em,nindent=.9em,slope=.6em]{#1}{#2}%---------------A
 }{%
  \ifthenelse{\equal{#1}{B}}{%
   \lettrine[lines=3,findent=-.4em]{#1}{#2}%---------------------------------------B
  }{%
   \ifthenelse{\equal{#1}{C}}{%
    \lettrine[lines=3,findent=-.2em]{#1}{#2}%--------------------------------------C
   }{%
    \ifthenelse{\equal{#1}{D}}{%
     \lettrine[lines=3,findent=-.4em]{#1}{#2}%-------------------------------------D
    }{%
     \ifthenelse{\equal{#1}{E}}{%
      \lettrine[lines=3,findent=-.2em]{#1}{#2}%------------------------------------E
     }{%
      \ifthenelse{\equal{#1}{G}}{%
       \lettrine[lines=3,findent=-.3em]{#1}{#2}%-----------------------------------G
      }{%
       \ifthenelse{\equal{#1}{I}}{%
        \lettrine[lines=3,findent=-.1em]{#1}{#2}%----------------------------------I
       }{%
        \ifthenelse{\equal{#1}{K}}{%
         \lettrine[lines=3,findent=+.1em]{#1}{#2}%---------------------------------K
        }{%
         \ifthenelse{\equal{#1}{L}}{%
          \lettrine[lines=3,findent=-1.1em,nindent=1.5em]{#1}{#2}%-----------------L
         }{%
          \ifthenelse{\equal{#1}{O}}{%
           \lettrine[lines=3,findent=-.3em]{#1}{#2}%-------------------------------O
          }{%
           \ifthenelse{\equal{#1}{Q}}{%
            \lettrine[lines=3,findent=-.3em]{#1}{#2}%------------------------------Q
           }{%
            \ifthenelse{\equal{#1}{R}}{%
             \lettrine[lines=3,findent=-.3em]{#1}{#2}%-----------------------------R
            }{%
             \ifthenelse{\equal{#1}{S}}{%
              \lettrine[lines=3,findent=-.2em]{#1}{#2}%----------------------------S
             }{%
              \ifthenelse{\equal{#1}{V}}{%
               \lettrine[lines=3,findent=+.1em,nindent=0em,slope=-.5em]{#1}{#2}%---V
              }{%
               \ifthenelse{\equal{#1}{W}}{%
                \lettrine[lines=3,nindent=-.1em,slope=-.4em]{#1}{#2}%--------------W
               }{%
                \ifthenelse{\equal{#1}{Y}}{%
                 \lettrine[lines=3,findent=+.1em,nindent=0em,slope=-.5em]{#1}{#2}%-Y
                }{%
                 \lettrine[lines=3]{#1}{#2}%---------------------------------------*
	        }%
               }%
	      }%
	     }%
            }%
           }%
          }%
         }%
        }%
       }%
      }%
     }%
    }%
   }%
  }%
 }%
}

%
% \LLimg[options]{filename}
%   Place an inline EPS image here in the text (this needs refinement)
%
\newcommand\LLimg[2][]{%
  \ifthenelse{\equal{#2}{}}{%
    \emph{XXX Image? #1 #2}%
    \typeout{Missing image name in LLimg!}%
  }{%
    \ifthenelse{\boolean{pdf}}{%
      \ifthenelse{\equal{#1}{}}{%
        \epsfig{file=images/#2, width=\textwidth}%
      }{%
        \epsfig{file=images/#2, #1}%
      }%
    }{%
      \ifthenelse{\equal{#1}{}}{%
        \epsfig{file=eps/#2.eps, width=\textwidth}%
      }{%
        \epsfig{file=eps/#2.eps, #1}%
      }%
    }%
  }%
}% 
%
%
% \URLref[*][text]{url}
%   Refer to a URL somehow (as <a href="url">text</a> would have been)
%   Starred form uses 2nd level of footnotes (for use within a footnote)
%
\newcommand\URLref[2][]{#1\footnote{\URL{#2}}}
\WithSuffix\newcommand\URLref*[2][]{#1\footnoteB{\URL{#2}}}
%
% \acronym{letters}
% \mc{letters}
%   Typesets all-caps in slightly smaller font.  \mc just does the
%   typesetting, while \acronym also adds an entry in the main index
%   for the term (maybe someday we'll add to the glossary as well).
%
\newcommand\acronym[1]{{\smaller #1}\index{#1@\acronym*{#1}}}
\WithSuffix\newcommand\acronym*[1]{{\smaller #1}}
\newcommand\mc[1]{{\smaller #1}}
%
% \starsep[symbol]
%   Star paragraph section separator.  The default symbol is
%   a fancy asterisk from the dingbats font.
%
\newcommand\starsep[1][\Asterisk]{
	\medskip
	\centerline{#1\qquad#1\qquad#1\qquad#1}
	\medskip
}

%
% \asidenote{text}
%   Side comments (just set in smaller type)
%
\newcommand\asidenote[1]{{\footnotesize #1}}

%
% unfinished
%
%\tracingmacros=1
\newcommand\strong[1]{{\bfseries #1}}
\newcommand\TM{\texttrademark}
\newcommand\work[1]{\emph{#1}\index{#1@\work*{#1}}}
\WithSuffix\newcommand\work*[1]{\emph{#1}}
\newcommand\URL[1]{{\ttfamily #1}}

%
% notes
%
% Within a group or environment:
%  switch font family:           \rmfamily \sffamily \ttfamily
%  switch series within family:  \mdseries \bfseries
%  switch letter shape:          \upshape \slshape \itshape \scshape
%
% (lower-level: select 12pt Garamond medium italic with T1 encoding with 15pt leading.)
%   \fontencoding{T1}
%   \fontfamily{garamond}
%   \fontseries{m}
%   \fontshape{it}
%   \fontsize{12}{15}
%   \selectfont
%
% relsize package: \smaller etc for relative sizing
% \begin{wrapfigure}[number of narrow lines]{placement}[overhang]{width} ... \end{wrapfigure}
%   placement: r=right, l=left, i=inside, o=outside,  <--here
%              R=right, L=left, I=inside, O=outside,  <--float
%   overhang: out into margin [default \wrapoverhang, initially 0]
%
% Indexing notes:
%	\index[indexname]{entry}
%	\ix{text} is convenience --> text\index{text}
%
%	As far as the entries themselves, the following special characters may be used:
%		main-entry[!sub-entry[!sub-sub-entry]]		(max 3 levels)
%		xxx|(		start page range for entry xxx
%		xxx|)		end page range for entry xxx
%		xxx|see{yyy}	make xxx cross-reference to yyy (place anywhere, not in doc)
%		xxx@yyy		sort as xxx, but typeset as yyy (may incl. TeX source in yyy)
%		xxx|fff		format entry xxx's page number with \fff{pageno}
%		xxx|fff(	start page range w/pageno formatting (end with "xxx|)")
%	
%	"@ "! "|  -> literal @ ! |
%
% if index doesn't show up in TOC, add this at index point:
% \clearpage (etc)
% \addcontentsline{toc}{chapter}{Index}
% \printindex
%
\newenvironment{NotImplemented}{\par{\bfseries The following describes a feature of the Lumos
controller which is planned for a future release but not implemented today.  While this description
follows the expected behavior the Lumos board will have when that feature is actually available,
it is still under development and subject to change.}
\par\begingroup\color{gray}}{\endgroup}
\newenvironment{NotImplemented*}[1]{\par{\bfseries #1}\par\begingroup\color{gray}}{\endgroup}
