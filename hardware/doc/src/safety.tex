\chapter{Safety Information}\label{ch:safety}

\LLstart{B}{efore}{you begin building} your Lumos controller, please take the time to
carefully read the following \ix{safety precautions}.  Failure to follow this advice could
result in death or serious injury, damage to the Lumos controller unit, and/or damage
to the other devices plugged into the controller.
\index{danger warnings}
\index{warnings}

\section{Hazardous Materials}
While assembling this unit you may come in contact with \ix{hazardous chemicals}.  The Lumos
product contains no hazardous parts \emph{per se} but if you choose to assemble it using
lead-based solder, you may expose yourself to risk of \ix{lead poisoning}.  The main cause of lead
\marginpar{\centerline{\LLimg[height=.5in]{poison_sign}}}
poisoning due to soldering is by ingesting lead particles left on your hands or work surface.
To avoid this, keep food away from your work area, and thoroughly wash your hands and work
surfaces before handling food.

In addition, if you choose to use other chemical agents (e.g., to clean excess flux from your
soldered board), be sure to read and follow their precautions and instructions carefully.

\section{Small Part Danger}
This board contains small parts which could pose a \ix{choking hazard} to small children.
This product is not a toy and is not intended for use by children in any circumstance.
The small parts on the product can be swallowed by children under 4~years of age. Keep
\marginpar{\centerline{\LLimg[height=.5in]{danger-generic}}}
out of reach of children.

\section{Hazardous Voltage}
Exercise care when working with any electrical system, including one such as the Lumos DC
controllers (even though in theory they deal with low voltages).  The power supplies of the
loads plugged into the Lumos controller, and even the power loads being controlled, may present
\marginpar{\centerline{\LLimg[width=.5in]{danger-shock}}}
a \ix{shock hazard} if not wired and handled using standard safety protocols.  Never touch or work
with live circuits. Always disconnect the power source before working on your Lumos controller.

When working with loads outdoors, be sure all supplies are plugged into \acronym{GFIC}-protected
circuits.

\section{Physical Hazards}
While assembling the unit, always wear \acronym{ANSI}-approved \ix{eye protection} gear.  When soldering,
always be aware of---and in control of---the location of your soldering iron (it will be 300$^\circ$F--500$^\circ$F---any slight 
\marginpar{\centerline{\LLimg[height=.5in]{danger-generic}}}
mistake can be costly, painful, or dangerous)! Bits of molten metal or flux can spatter onto your
skin or in your eyes during soldering.  When cutting leads, sharp metal wires may be launched into the
air, and could hit your eyes.

\section{Electrostatic Discharge (ESD) Warning}
Many of the components used in this project are sensitive to static electricity.  Always use a proper
\acronym{ESD}-safe work environment when handling them, or these parts may be permanently damaged.  If
a part is damaged in this way, 
\marginpar{\centerline{\LLimg[height=.5in]{esd_symbol_l}}}
it is impossible to tell by looking at the part, and you won't necessarily
feel the \ix{static discharge} which caused the damage.  Never take the risk of handling sensitive components
without \acronym{ESD} protection in place.
\index{danger warnings}
\index{warnings}

These parts include all transistors, voltage regulators, diodes (D0--D11),
and integrated circuits.
