\section*{NAME}
lumosupgrade -- Download new firmware image into Lumos controller hardware
\section*{SYNOPSIS}
\codetype{lumosupgrade}
[\codetype{--fhNnRv}]
\codetype{--a}
\Var*{addr}
[\codetype{--b}
\Var*{speed}]
[\codetype{--L}
\Var*{txlvl}]
[\codetype{--m}
\Var*{txdelay}]
[\codetype{--p}
\Var*{port}]
[\codetype{--T}
\Var*{mode}]
\Var*{image-file}
\section*{DESCRIPTION}


This command places the target Lumos controller device into
``flash program mode'' and downloads a new firmware image
onto it.  This is used in order to upgrade the Lumos firmware
to a new version.


To upgrade your controller's firmware, perform the following steps:
\begin{enumerate}
% tagged paragraph, width=3
\item
Connect your Lumos controller board to the host PC (all by itself,
not sharing a serial connection with other devices). The connection
must be bidirectional as the Lumos board needs to be able to acknowledge
receipt of the new image.  (This may require setting your PC's RS-485 
interface appropriately so it may send and receive data.  Pay attention
to whether your Lumos board is full- or half-duplex.)
% tagged paragraph, width=4
\item
Place the Lumos board into configuration mode (previously called
``privileged mode'').
% tagged paragraph, width=4
\item
Invoke the
\codetype{lumosupgrade}
program to load a new firmware image onto it, as explained in
the remainder of this manual page.
\end{enumerate}


Once started, this process must run to successful completion
before the Lumos board may be used for normal operations again.
If anything goes amiss, the Lumos board may be reset and the 
\codetype{lumosupgrade}
program restarted using the
\codetype{----resume}
option.  (Upon reset, the Lumos board will stay in
flash program mode until a new image has been loaded into it.  Should the
board reset/reboot for any reason during the process, the 
\codetype{lumosupgrade}
process must be started over to ensure a complete image is loaded.)
\section*{OPTIONS}


Each of the following options may be specified by either a long
option (like
``\codetype{----verbose}'')
or a shorter option letter (like
``\codetype{--v}'').
If an option takes a parameter, it may follow the option as
``\codetype{--a12}'',
``\codetype{--a 12}'',
``\codetype{----address 12}'',
or
``\codetype{----address=12}''.


Long option names may be abbreviated to any unambiguous initial substring.
\begin{list}{}{}
% tagged paragraph, width=19
\item[{\codetype{----address=}\Var*{addr}}]\hfill\\
(\codetype{--a}
\Var*{addr})
Specifies the address of the target controller unit.  The
\Var*{addr}
value is an integer from 0 to 15, inclusive.  
Note that downloading a new firmware image must be done when the
target unit is the
only
device plugged in to the computer.  Once the download operation is
underway, the low-level protocol used to transmit the image to the
device is not necessarily compatible with other units.  This option
is required because the command to place the device into flash program
mode must be addressed to the unit.
% tagged paragraph, width=4
\item[{\codetype{----dry--run}}]\hfill\\
(\codetype{--n})
Do everything except actually burn the new firmware into the unit.
This checks that the 
\Var*{image-file}
is reasonably sane-looking (not a thorough check of correctness),
and communicates with the Lumos controller up to the point where
it would put it into flash program mode.
% tagged paragraph, width=4
\item[{\codetype{----force}}]\hfill\\
(\codetype{--f})
Force upgrade of the board without asking the user for confirmation.
% tagged paragraph, width=4
\item[{\codetype{----help}}]\hfill\\
(\codetype{--h})
Print a summary of these options and exit.
% tagged paragraph, width=4
\item[{\codetype{----null--device}}]\hfill\\
(\codetype{--N})
Don't actually communicate with the serial port, but still carry out the other
actions including sanity checks on 
\Var*{image-file}.
Implies
\codetype{----dry--run}.
% tagged paragraph, width=4
\item[{\codetype{----port=}\Var*{dev}}]\hfill\\
(\codetype{--p}
\Var*{dev})
Specifies the I/O port the Lumos device is connected to.  This may be a simple integer 0, 1, 2, etc.
to refer to the first, second, third, etc, standard serial port on the system, or a device name appropriate
to the system such as 
\codetype{COM1},
\codetype{ttys1},
or 
\codetype{/dev/ttys1}.
% tagged paragraph, width=4
\item[{\codetype{----resume}}]\hfill\\
(\codetype{--R})
Indicates that the Lumos board was reset prematurely while attempting an upgrade.  Since the
Lumos board will still be in flash programming mode (and therefore won't
be in a position to recognize the Lumos-protocol command to begin a flash
programming operation), this option tells 
\codetype{lumosupgrade}
to simply start downloading the image onto it, and to not try to put it 
into programming mode first.
% tagged paragraph, width=4
\item[{\codetype{----speed=}\Var*{rate}}]\hfill\\
(\codetype{--b}
\Var*{rate})
Set the serial port to the given
baud
\Var*{rate}.
This is the speed the Lumos board is already configured to use, and will be used for the initial
command to enter flash programming mode.  Once in flash programming mode, however, a fixed speed of 9600
baud will be used.
[Default is 19200 baud.]
% tagged paragraph, width=4
\item[{\codetype{----txdelay=}\Var*{t}}]\hfill\\
(\codetype{--m}
\Var*{ms})
Delay
\Var*{ms}
milliseconds after changing the transmitter control line for half-duplex networks.
% tagged paragraph, width=4
\item[{\codetype{----txlevel=}\{\codetype{0}|\codetype{1}\}}]\hfill\\
(\codetype{--L}
\{\codetype{0}|\codetype{1}\})
Specifies the logic level used to signal transmit mode for half-duplex networks. A 
\codetype{1}
indicates that the DTR or RTS line (as selected by the
\codetype{----txmode}
option) is asserted to transmit, while a
\codetype{0}
means the line is deasserted to transmit.
% tagged paragraph, width=4
\item[{\codetype{----txmode=}\{\codetype{dtr}|\codetype{rts}\}}]\hfill\\
(\codetype{--T}
\{\codetype{dtr}|\codetype{rts}\})
Specifies which I/O line is used to signal transmit mode on half-duplex networks.
% tagged paragraph, width=4
\item[{\codetype{----verbose}}]\hfill\\
(\codetype{--v})
Output messages to the standard output.  Additional 
\codetype{----verbose}
options increase verbosity.  High levels of verbosity include a dump
of every bit sent or received on the serial network.
\end{list}
\section*{FILE FORMAT}


The firmware
\Var*{image-file}
is expected to be in standard Intel Hex format.  Attempts to change memory
addresses outside the supported range will be ignored, including configuration
fuses and EEPROM area.  Actually, only a reasonable subset of the Intel Hex
format is supported; specifically, record types
\codetype{00}
(data record),
\codetype{01}
(end of file),
and
\codetype{04}
(extended address)
are recognized.
\section*{AUTHOR}


Steve Willoughby,
Software Alchemy / support@alchemy.com
\section*{COMPATIBILITY}


This version of 
\codetype{lumosupgrade}
is compatible with the following boards:
\begin{itemize}
% tagged paragraph, width=2
\item
Lumos 24-channel DC controller version 1.0 (boards with ID markings beginning with
``LUMOS-24SSR-DC-1.0'').
% tagged paragraph, width=4
\item
Lumos 4-channel DC controller version 1.0 (boards with ID markings beginning with
``LUMOS-4SSR-DC-1.0'').
\end{itemize}


\emph{The 48-channel controllers are not compatible with this program.}
\emph{These boards must be reprogrammed using a microcontroller programmer.}
\section*{HISTORY}


This program first appeared to support Lumos ROM version 3.0.
\section*{SEE ALSO}
\codetype{lumosctl}(1).
