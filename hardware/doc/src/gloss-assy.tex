\chapter{Glossary}\label{ch:glossary}
\begin{description}
	\item[Active High:]
		A logic signal which is considered ``on'' when the signal is ``high'' (binary 1 or +5\,V),
		and ``off'' when the signal is ``low'' (binary 0 or 0\,V).  Lumos relay circuits are 
		triggered with active-low signals.
	\item[Active Low:]
		A logic signal which is considered ``on'' and ``off'' at the opposite signal levels
		to an ``active high'' signal (q.v.).
	\item[Annular Ring:]
		The exposed ring of metal around a hole in a \acronym{PCB} where a component is to be 
		mounted.  The solder will flow across the component lead and onto the annular ring.
%	\item[Daisy Chain:]
%		The arrangement of wiring a number of devices together by connecting the first to the second,
%		then adding another connection from the second to the third, and so forth.  The network
%		connection diagram in Figure~\ref{fig:net} shows an example of a daisy chain.
	\item[\acronym{DIP} (Dual In-line Package):]
		The style of chip where the pins are laid out in two parallel rows.
	\item[\acronym{DIY}:] ``Do-It-Yourself.''
	\item[Duplex:]
		a feature of a serial line.  On a full-duplex connection, separate data wires are present
		to carry data in both directions, so one device can send and receive data at the same time.
		On a half-duplex connection, only a single set of data wires is present, so devices must
		take turns transmitting over them.
	\item[\acronym{ESD} (Electro-Static Discharge):]
		static electricity which builds on your skin and is then discharged into sensitive
		components when you touch them.  Invisible to the eye, this can punch microscopic holes
		in the inside of the components, severely damaging them.
	\item[Heat Protection:]
		A temporary heat sink applied to a component when soldering that component onto
		the \acronym{PCB}.  Typically used for heat-sens\-i\-tive components such as transistors
		and integrated circuit chips.
	\item[Jumper Block:]
		A series of pins mounted to the \acronym{PCB}.  Different options are configured for the
		circuit by placing a jumper over certain pairs of pins, shorting them together.
	\item[\acronym{LED} (Light Emitting Diode):]
		A special kind of diode which emits light when current passes from its anode to its cathode.
	\item[\acronym{MOSFET}:]
		The type of transistor which forms the major part of a Lumos DC relay channel.  The name
		is an acronym for Metal Oxide Semiconductor Field Effect Transistor.
	\item[\acronym{PCB} (Printed Circuit Board):]
		The board where electronic components are mounted to form a complete circuit.  Metal
		traces are ``printed'' (actually etched) onto the surface of the board itself to make the
		connections between components.
	\item[RS-232:]
		A standard hardware protocol for sending serial data between two devices (such as a computer
		and a modem or a single Lumos board).  Shielded cable should be used for best results, and
		the cable length should not exceed 25\,ft.
	\item[RS-485:]
		A standard hardware protocol for sending serial data between multiple devices on a single
		cable length (electrically it is a single cable which each device ``taps into'' along the
		line; physically it is typically a ``daisy chain'' arrangement where a short cable connects
		one device to the next, another cable to the next, and so on). Unshielded twisted-pair cable
		is used (like Ethernet cable), and the cable lengths should not exceed a total of 4,000\,ft
		(1,200\,m).
	\item[Terminator Plug:]
		An \acronym{RS-485} network requires a terminator at each end.  This is a small plug which plugs into
		the last unit in the daisy chain.
	\item[\acronym{TTL} (Transistor-Transistor Logic):] One of the ways digital logic circuits can be
		constructed.  For our purposes here, we consider a ``\acronym{TTL}-level'' signal to be a
		logic input or output where a voltage near +5\,V is ``high'' (binary 1 or ``true'') and a
		voltage near 0\,V is ``low'' (binary 0 or ``false'').  The inputs should never be above
		+5 nor below 0 volts.
\end{description}
