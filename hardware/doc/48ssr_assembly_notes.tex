>B 1000aaaa all channels off
>  1001aaaa 0scccccc           Set channel c off (s=0) or on (s=1)
>  1010aaaa 00cccccc 0vvvvvvv  Set channel c to level vvvvvvvv (0--127)
>  1010aaaa 0hcccccc 0vvvvvvv  Set channel c to level vvvvvvvvh (0--255)
>  1011aaaa 00cccccc 0nnnnnnn xn 01010101  Update channel c--(c+n+1) to level v (0-127)
>  1011aaaa 01cccccc 0nnnnnnn xn 01010110 0hhhhhhh x (n+1)/7 01010101  Update channel c--(c+n+1) to level hvvvvvv (0-255)
>  1100aaaa 0dcccccc 0sssssss 0ttttttt  Ramp (d: 0=dn, 1=up) in s+1 steps with (t+1)/120 sec/step
>  1101aaaa				Reserved 
>  1110aaaa				Reserved 
AB>1111aaaa 00000000 01011010 01011010	Enter sleep mode
AB>1111aaaa 00000001 01011010 01011010  Exit sleep mode
>B 1111aaaa 00000010 01011000 01011001  Shutdown device completely
 R 1111aaaa 00000011 00100100 01010100  Report device status
>  1111aaaa 00000100 0iiiiiii 0nnnnnnn ... 01000100 01110011  Define sequence i
					   of length n; i=0 is boot sequence;
					   i<64 permanently stored if possible
>  1111aaaa 00000101 0iiiiiii           Execute sequence i; i=0 to stop current
>  1111aaaa 00000110 0owm00ss 0iiiiiii 0ppppppp 0ttttttt 00111100
					  Define sensor trigger when sensor(s)
                                          ss: 00=a, 01=b, 10=c, 11=d;
                                          m: 0=active low 1=active high
					  activate (abcd 1=selected);
					  i is played to initialize,
					  p during the event,
					  t at termination
					  trigger once if o=1, else while
					  sensor is high if w=1, else loop
					  forever once activated; blocks
					  other channel-setting commands if
					  b=1  ## NO: USE COMMAND IN SEQ INSTEAD
>  1111aaaa 00000111 0000ABCD           Mask sensor triggers (1=enable, 0=dis)
>  1111aaaa 00001000 01000011 01000001  Clear all stored sequences 
>  1111aaaa 00001001                    Reserved non-admin extended command
      :                                   :
>  1111aaaa 00011110                    Reserved non-admin extended command
>  1111aaaa 00011111                    Reserved for internal use
>  1111aaaa 001xxxxx                    Reserved for internal use
PC>1111aaaa 010000pp 0ppppppp 01010000 01001111 set phase offset (0-511)
PC>1111aaaa 0110aaaa 01001001 01000001 01000100 Set device address (0-15)
BC>1111aaaa 01110000                            Disable configuration mode
PC>1111aaaa 01110001 0ABCDdcc 0ccccccc 0r111010 00111101 Set configuration
                       ABCD (0=outputs, 1=inputs)
                       d (DMX512 0=off, 1=on; channel 0=DMX c+1)
                       r (resolution 0=low, 1=high)
>C 1111aaaa 01110010 0bbbbbbb 00100110  Set baud rate
>C 1111aaaa 01110011 00100100 01110010  Restore factory settings       	XXX CHANGED XXX
 C 1111aaaa 01110100                    Reserved admin command
      :                                   :
 C 1111aaaa 01111111                    Reserved admin command



A - automatic
B - broadcast
C - configuration mode
P - permanent change
R - reply expected

Status reply packet:
1111aaaa 00011111 <data> 00110011
<data>::=
  0ABCDdcc 0ccccccc 0ABCDqsf 0ABCDrpp 0ppppppp 0eeeeee 0eeeeeee 0mmmmmmm 0mmmmmmm 0X0000ii 0xxxxxxx
   \__/|\_________/  \__/|||  \__/|\_________/ \______________/  \______________/  |    \/ \______/
   |   |     |         | |||    | |       |             |                 |        |     |     |
sens   |     |         | |||    | |       |             |                 |        |     |     |
DMX----+     |         | |||    | |       |             |                 |        |     |     |
DMX channel--+         | |||    | |       |             |                 |        |     |     |
sensor masks (1=en)----+ |||    | |       |             |                 |        |     |     |
in admin/config mode now-+||    | |       |             |                 |        |     |     |
sleep mode (s=1)----------+|    | |       |             |                 |        |     |     |
last seq overfilled--------+    | |       |             |                 |        |     |     |
Current sensor input states-----+ |       |             |                 |        |     |     |
Resolution (1=high, 0=low)--------+       |             |                 |        |     |     |
phase offset value------------------------+             |                 |        |     |     |
EEPROM bytes free for sequences-------------------------+                 |        |     |     |
RAM bytes free for sequences----------------------------------------------+        |     |     |
A sequence is currently executing (1)----------------------------------------------+     |     |
Device Model ID (00=48SSR, 01=24SSR-DC)--------------------------------------------------+     |
Currently running sequence---------------------------------------------------------------------+
   0owmxx00 0iiiiiii 0ppppppp 0ttttttt 	sensor config for A
   0owmxx01 0iiiiiii 0ppppppp 0ttttttt 	sensor config for B
   0owmxx10 0iiiiiii 0ppppppp 0ttttttt 	sensor config for C
   0owmxx11 0iiiiiii 0ppppppp 0ttttttt 	sensor config for D


SEQUENCE DATA
EXECUTION MODEL

Call/return stack	xx levels of exec index bytes with pointer
Parameter stack		??
Loop frame stack	xx levels of counter variables
Registers		4 registers available

TRANSMITTED				SOURCE			COMPILED

00000000				exit
0ri00001 0scccccc			channel <c> on|off	
0ri00010 0vcccccc 0vvvvvvv		channel <c> <v>[%]
00000011				// bulk??
0ri00100 0dcccccc 0sssssss 0ttttttt	ramp up|down <c> [by <s+1>] [per <t+1>]
00000101				// reserved
00000110				// reserved
00000111				sleep
00001000				wake
00i01001 0iiiiiii            		execute <i>
00i01010 0iiiiiii p0 p1 p2 p3		execute <i> (p0, ... pn)
00i01011 0iiiiiii			call <i>
00i01100 0iiiiiii p0 p1 p2 p3      	call <i> (p0, ..., pn)
0ri01101 0ttttttt			wait <t+1>   // 1/120 sec
0ri01110 0sssssss 			for <var>=<start> to <end> [by <step>]: 
0ri01111 0sssssss 0eeeeeee 0aaaaaaaa	... end
00010000 0n00ABCD 0aaaaaaa		if|unless sensor a|b|c|d [or a|b|c|d ...]: ... end
0ri10000 0nOO0--- 0aaaaaaa              if|unless <var|param|const> {<|>|=} <var|param|cons>: ... end  OO=01< 10> 11=
00010000 0n111--- 0aaaaaaa              if|unless suspended
00010001				blackout
0f010100				suspend [with update]  	// stop executing incoming commands; save current state of SSRs; continue updating saved state if f=1; ignored if already on
0f010101				resume [with fade]	// executing incoming commands, restore saved state; fade through black if f=1; ignored if already resumed

					NO SOURCE; OPCODES GENERATED BY COMPILER

00010010 0aaaaaaa			branch <bytes>
00010011 0000wxyz			clear register(s) w|x|y|z
00010011 0oo000rr 0vvvvvvv		oo immediate v -> reg rr     oo: 00=load 01=add 10=sub 11=mul
01110011 0ookkkrr 			oo loop ctr kkk -> reg rr
00110011 0ookkkrr 			oo param kkk -> reg rr



					PSEUDO-OPS

        				repeat [<n>] [as <var>]: ... end   -->   for <var>=1 to <n>:   repeat:  == forever
					sequence <i> [(p0, ..., pn)]: ... end
					break <loop idx>
					continue <loop idx>
					label:
                                        
a = address of target instruction (absolute byte position within sequence):
c = channel number
i = 0=immediate args, 1=use register(s)
r = 0=absolute channel, 1=add current loop ctr
s = 0=off, 1=on


// Sequence 1 - ramp up all channels to full brightness
sequence 1:				
	for x=0 to 24:			00: 0E 00
		ramp up x by 2		02: 44 00 01 00
		wait 60			06: 0D 3B
	end 				08: 0F 00 17 02
end					0C: 00


To program this into controller #0, send:
F0 04 01 0D 0E 00 44 00 01 00 0D 3B 0F 00 17 02 00 44 73

To execute it on controller #0, send:
F0 05 01


sensor input A triggers sequence 2 when started, then 3 when ended
sensor input B triggers sequence 4 when started, then 5 when ended.

None of these (3, 5) fire when either sensor is still active, though.
sequence 2:
	unless suspended:			10 78 15
		suspend with updates		54
		for c=0 to 21 by 3:		0E 00
			ramp up c+0 by 2	44 00 01 00
			ramp up c+1 by 2	44 01 01 00
			ramp up c+2 by 2	44 02 01 00
			wait 60			0D 3B
		end				0F 00 15 11
	end					
end						00

sequence 3:
	if suspended:				
		for c=21 to 0 by 3:		
			ramp down c+0 by 2
			ramp down c+1 by 2
			ramp down c+2 by 2
		end
		resume with fade
	end
end

sequence 4:
	if not suspended:
		suspend with updates
		for c=21 to 0 by 3:
			ramp up c+0 by 2
			ramp up c+1 by 2
			ramp up c+2 by 2
			wait 60
		end
	end
end

sequence 5:
	if suspended:
		for c=0 to 21 by 3:
			ramp down c+0 by 2
			ramp down c+1 by 2
			ramp down c+2 by 2
		end
		resume with fade
	end
end
	


transmitted           source            compiled
00000000              kill		11111100 		all off
0r000001 x	      set c [on/off]	0scccccc		set on/off
0r000010 x x 	      set c x%		10cccccc vvvvvvvv	set value
0r000011 ...	      set ...		11111000 nnnnnnnn v+    set bulk
0r000100 x x x	      fade x x x	11111001 x x x          ramp         r=0 absolute channel #; r=1 added to "for" loop index
00000101 
00000110
00000111              sleep		11110000		sleep
00001000	      wake		11110001		wake
00001001 i	      run i		11110010 i		exec i
00001010 n	      repeat n:		11110011 n s C		rept n -s
00001011 i	      call i		11110100 i		call i
00001100 x y s        end (for)         11110101 x y s          loop +x to y -s
00001101 00obABCD s   if ...		11110110 f s		if f -s
00001110 m	      wait m		11110111 m		wait m/120 sec
00001111 	      stop		11111111		exit
00010000 x	      for chan x:y[:s]	11111010 x		set c=x


blackout
set [+]c on|off|x% [, ...]
fade [+]c up|down steps time
sleep
wake
goto i
call i
repeat [n [times]]: ... end
for start to end [by step]: ... end
if [not] input A|B|C|D: ... end
stop

for 0 to 21 by 3:
   fade +0 up 4 1 	// +0 means loop channel + offset  XXX make that a flag in the command
   fade +1 up 4 1
   fade +2 up 4 1
   wait 0.25
end

compiles to:

HEX
10 00		FOR 0 to 23 by 3:
44 40 03 00     FADE UP ch+0 4 1
44 41 03 00     FADE UP ch+1 4 1
44 42 03 00     FADE UP ch+2 4 1
0E 1E		WAIT 30/120 sec
0C 03 15 14     END FOR (Skip back -0x14 bytes to start of loop)

Sent by: (if sequence #1 on unit #0)

F0 04 01 14 10 00 44 40 03 00 44 41 03 00 44 42 03 00 0E 1E 0C 03 15 14 44 73
defsq #1 len 0  1  2  3  4  5  6  7  8  9  A  B  C  D  E  F 10 11 12 13 sent.
            <-----------------------payload---------------------------> <--->

for 21 to 0 by 3:
   fade +0 up 4 1 	// +0 means loop channel + offset  XXX make that a flag in the command
   fade +1 up 4 1
   fade +2 up 4 1
   wait 0.25
end

10 15
44 00 03 00
44 01 03 00
44 02 03 00
0E 1E
0C 03 00 14





%
% vi:set ts=8 sw=8 ai sm wm=3:
% Lumos 48-Channel SSR Controller 
% Assembly Notes
% Steve Willoughby
%
%
\documentclass[12pt]{article}
\hyphenation{EEPROM}
\newcommand\mc[1]{{\small #1}}
\usepackage{bytefield}
\usepackage{longtable}
\usepackage{epsfig}
\begin{document}
\title{Lumos SSR Assembly and Usage Guide}
\author{Steve Willoughby \\ for AC controller version 4.0 \\ and DC controller version 1.0.1 \\ (Firmware revision 4.0)}
\date{31 July 2012}
\maketitle
\tableofcontents
\begin{center}
\begin{tabular}{|p{4.5in}|}\hline
{\large\bf\strut SAFETY WARNING!}\\\hline
\bf \mc{DANGER!  RISK OF FIRE, ELECTROCUTION, SERIOUS    
INJURY OR DEATH!}  This circuit design, including 
but not limited to any accompanying plans,       
designs, board layouts, documentation, and/or    
components, is \mc{EXPERIMENTAL} and for \mc{EDUCATIONAL}
purposes only.  It is not a finished consumer-grade product
and has not been reviewed by a body such as Underwriter's
Laboratories.  It is assumed that you have the necessary 
understanding and skill to assemble and/or use electronic
circuits.

Proceed \mc{ONLY} if you know exactly what you are    
doing, understand the proper procedures for      
working with the high voltage present on the     
components and PC boards, and understand that    
you do so \mc{ENTIRELY AT YOUR OWN RISK.}

The author makes \mc{NO}
representation as to suitability or fitness for  
any purpose whatsoever, and disclaims any and all    
liability or warranty to the full extent permitted
by applicable law.\\\hline
\end{tabular}
\end{center}

If you got this far, after reading the above safety warning, the rest of this
document contains some useful but somewhat random notes for assembling, using
and maintaining the 48-channel SSR unit.  (If you skipped over the safety 
warning to just start reading this, {\em go back\/} and read it.  I don't want anyone
getting hurt by experimenting with an idea-in-progress like this.)

{\bf Please read this document completely before you begin construction.}

\newpage

\section{Introduction and Overview}
This document is simply a collection of notes I'm maintaining about my
experiments at computer-controlled Christmas light displays.  Like many
people, I think it would be entertaining to set up large
displays of lights on my front lawn and synchronize them to music, or at
least have them change patterns and color schemes at various times
throughout the night.

To this end, I have designed a controller which will manage up to 48
circuits of small 120\,V AC loads (like strings of incandescant Christmas 
lights) under the control of a computer system.  The computer sends
commands on a serial line to the controller box, which responds by
making the requested circuit changes.

In the sections that follow, the various components of this controller
are described, along with notes on assembling them from scratch and how
to install and use them.
 
Up to 16 of these units may be ``daisy-chained'' together to control a
total of 768 light circuits from a single serial port.

\section{Version Information}
This document describes the 48-channel SSR Controller board version 3.1
(board part markings begin with ``48CTL-3-1'') and the 24-channel SSR Relay
board version 3.0 (board part markings begin with ``24SSR-3-0'').

The current firmware revision as of this writing is 2.0.


\section{Required Supporting Items}
In order to actually use one or more of these controllers, you will
need the following supporting items:
\begin{itemize}
\item
{\bf Host Computer.}
The computer which runs the show doesn't really need to have any 
particularly spectacular capabilities, other than a serial port (if you
don't have a serial port, you can get an inexpensive USB-to-serial
adapter).  I've run my entire display on a small laptop before.
\item
{\bf Cabling.}
The cables connecting the controller(s) to the host computer are, at a
minimum, unshielded twisted-pair wire.  CAT3 telephone wire is probably
sufficient for our purposes.  The controller uses a single pair to
communicate with the host, plus a ground reference wire.

The connectors on the controller are 8-conductor RJ-45 modular jacks,
with pins 4 and 5 carrying data, and 7 and 8 both grounded.  This configuration 
allows standard Ethernet cables to be used, and has the added benefit
of using the two pair that 10- and 100-base-T Ethernet {\em does not\/}
use.  Note that the cabling must be {\em straight-through\/} and not
``crossover'' configuration.

The exact pinout of each RJ-45 jack is:

\begin{center}
\begin{tabular}{|c|l|}\hline
\bf Pin & \bf Signal \\ \hline\hline
4 & RS-485 data ``B'' ($-$) \\\hline
5 & RS-485 data ``A'' (+) \\\hline
7 & Ground \\\hline
8 & Ground \\\hline
\end{tabular}
\end{center}

\item
{\bf Power.}
You will obviously need a source of power to run the controlled 
circuits and the controllers themselves.  Each unit requires
one 120V plug for the controller, plus one 120V plug for each block of 8
circuits being controlled.

Be careful to check the current rating (amps) of the loads you are
attaching to the controller and make sure they do not exceed the
capacity of the controller {\em and\/} that the capacity of the power
supply it's all plugged in to is not exceeded.

\item
{\bf Software.}
You'll need some kind of software to control the lights.  The low-level
protocols used by the controller are documented here.  I have a simple
Python script I use to send the correct sequence of commands to the
boards in real-time.

I'm working on a full show control application as well, but this is 
still in the initial stages of development.

\item
{\bf RS-485 Converter.}
The serial communication signals used by these controllers are RS-485
standard signals.  Most computers use RS-232 signals, so you will need
an RS-232 to RS-485 converter.  You may also need some kind of adapter
to get the proper pinout for the controllers.

The first cable will run from your (terminated) converter to the first
controller.  This plugs into the ``RS-485 IN'' jack.

The next cable goes from the first controller's ``RS-485 OUT'' to the 
second controller's ``RS-485 IN,'' and so on.  

The ``RS-485 OUT'' jack on the last controller must have a terminator
plugged in to it.

(Incidentally, there is no difference between the RJ-45 jacks.  You need
to put the input signal on one, and either a terminator or the output
to the next unit on the other, but it really doesn't matter which is
which.)
\end{itemize}

\section{Relay Board Assembly}
Please read this {\bf entire} section {\bf before} beginning to assemble
the unit.

Each relay board contains 24 solid-state relays (SSRs) capable of
switching on, off, or dimming 120V loads of up to 5A each.  These SSRs
are arranged into blocks of 8, each block having its own separate power
supply and 8A fuse.

Each SSR consists mainly of a triac which is optically isolated from
the logic circuit driving it.  These are negative-logic based, so we
supply a +5V power signal to every opto-isolator (U$x$) on the board,
and the controller is connected such that pulling the SSR's control pin
to ground will activate the relay, and tri-stating it or pushing it up
to +5V will deactivate the relay, which will shut off at the next AC line
0V crossing point.

\emph{Exercise caution when the board is powered on!}  Live 120\,V AC mains 
current is active on several parts of the PC board when power is
applied.  These include many of the traces on either side of the
board, terminal blocks, triacs, heat sinks, triac mounting hardware,
optoisolator pins, fuses, etc.

\subsection{Bill of Materials}
Check that you have the following parts on hand.  In the following list
and the text which follows, $x$ in a part ID is the SSR number that
part belongs to, where $0 \le x \le 23$.
 \begin{longtable}[c]{|r|c|p{2.25in}|l|}\hline
  \bf Qty & \bf ID & \bf Value/Description & \bf Digi-Key\\\hline\hline
  \endhead
   1 & D0     & LED, T1-3/4 (5mm), Green   & P375-ND  \\\hline
   3 & F0--F2 & Fuse, 8A, 3AG glass, slo-blo& F2549-ND \\\hline
   3 &        & Fuse holder, 3AG type      & F1498-ND \\\hline
   1 & J0     & 26-pin ribbon cable header & MHB26K-ND\\\hline
   6 & J1--J6 & 10-position terminal strip & ED2225-ND\\\hline
   1 & R0     & $220\Omega$ resistor, 1/4W & P220BACT-ND\\\hline
  24 & R$x$0  & $560\Omega$ resistor, 1/4W & P560BACT-ND\\\hline
  24 & R$x$1  & $100\Omega$ resistor, 1/4W & P100BACT-ND  \\\hline
  24 & TR$x$  & Triac, 16A, {\em non-insulated\/} tab & MAC16DOS-ND\\\hline
  24 &        & TO-220 heat sink, horizontal & 345-1023-ND\\\hline
  24 & U$x$   & MOC3012 optoisolator, triac output, random phase, 5mA
  emitter current &\\\hline
 \end{longtable}

The Digi-Key part numbers are given as an example for comparison 
purposes only.  The parts will likely be available elsewhere, but
if you substitute parts, be sure they are the same physical size
as the specific Digi-Key parts, because the PC board is sized to
accommodate the parts specified above.

Additionally, you will need the power cables to bring power to and from
the relay units, and appropriate enclosures and mounting hardware, heat
sink grease, etc.
\subsection{Assembly Notes}
The board has fairly tight component spacing, so start with the
smaller, flatter components and move up to the bigger ones.
\begin{enumerate}
\item In the run of PC boards with part numbers beginning with ``24SSR-3-0'', 
      the mounting holes for the terminal blocks J1--J6 are small enough to 
      make for a very tight fit.  In fact, they may be just barely tight enough
      that you could have difficulty inserting the terminal blocks into the holes
      without damaging them.  We recommend drilling out these holes \emph{just
      ever so very slightly} (we're talking a couple thousandths of an inch if you can
      manage it).  When I did this, I used a tiny carbide drill bit held in my bare
      fingers, twisted it into the holes, and turned it a few turns.  The holes
      should not be enlarged more than a very few mils.  {\bf Do not drill} the holes
      which connect to a component-side (``top'') copper trace.  This is the ``hot''
      AC supply (there is one for every two terminal strips).  Those holes {\bf must}
      keep their inner plating.
\item Install the opto-isolators U0--U23.  Carefully solder in place, taking
      care to properly heat-sink the pins as you solder them.
\item Install the resistors R0, R$x$0 and R$x$1.
\item Install the terminal blocks J1--J6 so that the terminals face outward
      toward the outer edge of the PCB.
\item Install the 26-pin ribbon cable header connector (J0).
\item Install the LED D0.
\item Install the triacs TR0--TR23.  
      Be sure to apply enough heat sink grease to
      make good thermal contact, without spilling outside the contact area or
      onto the tab/bolt area.  Attach to heat sink, and secure both to the 
      board with a \#6 bolt and nut
      (this is needed for good electrical contact between the tab and the
      circuit board).  Solder the pins of the triac onto the board, taking
      care to properly heat sink them.  Ensure the bolt is fastened securely
      and making solid electrical contact between the triac tab and the 
      solder-side PC board pad.
\item Install the fuse holders F0--F2.
\item Place 8A fuses into the fuse holders F0--F2.
\item Use insulated stand-offs and appropriate machine bolts, washers and
      nuts to mount the relay PCB into its enclosure.
\end{enumerate}

\section{Controller Board Assembly}
Please read this {\bf entire} section {\bf before} beginning to assemble
the unit.

Each controller board is capable of controlling two 24-channel relay
boards, for a total of 48 channels.

The controller board is functionally arranged as a pair of 24-channel
controllers working in concert.  Each set of 24 channels is controlled
by one of the two microcontrollers U2 and U3.  U2 is designated the
``master'' controller---it receives and interprets all the commands
from the serial network, acting directly on commands affecting channels
0--23.  Any commands for the other 24 channels are passed on to the
``slave'' controller (U3).

Theoretically, the master controller may ask the slave to transmit
data back onto the serial network (U3 controls the data flow direction
of the RS-485 bus transceiver as well as driving the transmitter), but
in the current version of the controller firmware, this never actually
happens.  

The controllers send their output signals to the 24-channel relay boards
via 26-position ribbon cables.  These are negative-logic signals; a 0 (0V)
signal indicates that the corresponding SSR should be on, while a 1 (+5V) 
indicates that it should be off.  The relays are designed so that the SSRs
``float'' to off if no current is on the output lines at all.  When a 
controller is first powered on (and possibly at other times), the output lines
of the microcontrollers will be at hi-Z (no current drive either direction).
Also, one of the SSR drive signals on each microcontroller is driven by an
open-drain I/O pin, so it will always sink the line to ground to activate the
relay, or go to high-Z (instead of +5V) to deactivate it.

The controller contains its own integrated power supply, so that it can
monitor the 120V mains power via the zero-crossing detector composed of
R1, R2, R14 and U1.

\emph{Exercise caution when the board is powered on!}  Live 120\,V AC mains 
current is active on several parts of the PC board when power is
applied.  These include some of the traces on either side of the
board, terminal blocks, transformer pins, optoisolator pins, fuses, 
other components, etc.

\subsection{Bill of Materials}
Check your parts against the bill of materials to ensure that you
have all the right parts.
 \begin{longtable}[c]{|r|c|p{2.25in}|l|}\hline
  \bf Qty & \bf ID     & \bf Value/Description & \bf Digi-Key\\\hline\hline
  \endhead
  1 & C0               & $1800\mu$F electrolytic capacitor & P5587-ND* \\\hline
  1 & C1               & $0.33\mu$F capacitor & P4959-ND\\\hline
  3 & C2,3,8           & $0.1\mu$F capacitor & BC1084CT-ND\\\hline
  4 & C4--7            & 22pF capacitor & 399-1926-ND\\\hline
  3 & C9--11           & $0.01\mu$F capacitor & BC1078CT-ND\\\hline
  1 & D0               & Rectifier bridge, 2A, 200V & 2KBP02M/1GI-ND\\\hline
  3 & D1,3,6           & LED, T1-3/4 (5mm), Green   & P375-ND* \\\hline
  3 & D2,4,7           & LED, T1-3/4 (5mm), Amber   & P372-ND* \\\hline
  2 & D5,8             & LED, T1-3/4 (5mm), Red     & P582-ND* \\\hline
  1 & F0               & Fuse, 3AG, 120V 1/4A       & F2502-ND \\\hline
  1 &                  & Fuse holder, 3AG type      & F1498-ND \\\hline
  1 & J0               & 4-position terminal strip  & ED2221-ND\\\hline
  2 & J1,2             & 26-position box header &  MHB26K-ND\\\hline
  2 & J3,4             & RJ-45 jack, 8/8 pins & A31407-ND\\\hline
  2 & J5,6             & 5-position header & WM8087-ND\\\hline
  1 & P0               & 120\,V AC line cord \& plug & \\\hline
  8 & R0,4,6--8,11--13 & $220\Omega$ resistor, 1/4W & P220BACT-ND*\\\hline
  2 & R1,2             & 15K resistor, 1/2W & P15KBBCT-ND\\\hline
  2 & R3,9             & 33K resistor, 1/4W & P33KBACT-ND\\\hline
  2 & R5,10            & $330\Omega$ resistor, 1/4W & P330BACT-ND\\\hline
  1 & R14              & 27K resistor, 1/4W & P27KBACT-ND\\\hline
  1 & S0               & SPST 120V power switch\dag & \\\hline
  1 & T0               & Transformer, 120V pri, 8V sec, 1.25A, PCB mount & MT3115-ND \\\hline
  1 & U0               & 7805 +5V voltage regulator, 1.5A & LM7805CT-ND \\\hline
  1 & U1               & H11AA1 opto-isolator, AC input, NPN output & H11AA1300-ND\\\hline
  2 & U2,3             & PIC16F877A microcontroller & PIC16F877A-I/P-ND\\\hline
  1 & U4               & SN75176 RS-485 line driver/receiver & 296-1737-5-ND\\\hline
  2 & X0,1             & 20MHz crystal & CTX062-ND \\\hline
  1 &                  & TO-220 heat sink, horizontal & 345-1023-ND\\\hline
  1 &                  & 8-pin DIP socket & AE10011-ND\\\hline
  2 &                  & 40-pin DIP socket & AE10018-ND\\\hline
  2 &                  & \#4 machine screw (1.5") & \\\hline
  2 &                  & \#4 nuts & \\\hline
  2 &                  & \#4 1/4" insulated standoffs & \\\hline
  2 &                  & \#6 machine screw (3/8") & \\\hline
  2 &                  & \#6 lock washer & \\\hline
  2 &                  & \#6 nut & \\\hline
  4 &                  & sets of mounting hardware to secure the board
                         into its enclosure: \#8 machine screw, \#8 nut,
			 appropriately sized insulated standoffs, washers,
			 etc. & \\\hline
\end{longtable}

\noindent * indicates parts which I obtained elsewhere but I believe this
Digi-Key part to be equivalent.\\
\dag\ indicates an optional part.

As with the previous parts list, 
the Digi-Key part numbers are given as an example for comparison 
purposes only.  The parts will likely be available elsewhere, but
if you substitute parts, be sure they are the same physical size
as the specific Digi-Key parts, because the PC board is sized to
accommodate the parts specified above.

If you wish to provide a way to reset the unit without powering it off/on,
plug a switch of some sort on pins 2 and 3 of J5 and J6.  When these two 
pins are shorted on one of these jumper blocks, the corresponding CPU is
halted and reset.  When the contact is broken, the CPU will start running
from the start of its program again, just as when power is first applied.

\subsection{Assembly Notes}
Some of the components on this board (notably the transformer) are tall, and
there are some areas (around the CPUs and ribbon cable connectors) which are
somewhat tight in the component placement.  In consideration of this, I
recommend the following sequence for assembly of this board:
\begin{enumerate}
 \item Locate and solder the IC sockets first.
 \item In the run of PC boards with part numbers beginning with ``48CTL-3-1'',
       the mounting holes for resistors R1 and R2 are too small to fit a 1/2 watt
       resistor.  If your resistors don't quite fit, it will be necessary to slightly
       drill out those four holes.  They are currently 20~mils (0.5~mm) and they
       should be between 39--42~mils (approx.\ 1~mm).  Since these only connect to
       traces on the solder side (``bottom'') of the PC board, it is not a problem 
       to drill out the through-hole plating, but beware of destroying too much
       of the copper pad around the hole.  
 \item Install R1 and R2 into those holes and solder on the solder side (``bottom'') 
       of the board.  Test to make sure you have solid connections and continuity from
       each of the resistor leads to the parts to which they connect.
 \item Next, solder the remaining resistors R0--R14.  Pay attention to the placement of
       R3 and R9, so that they are out of the way of the sockets and the IC
       chips.  Resistors R5 and R10 place a series resistance between the CPU
       and crystal to protect against overdriving the oscillator.  
       This may not be needed (in which case you can replace
       those resistors with wire jumpers), but I've had success putting 
       $330\Omega$ resistors there, so try that first.
 \item Place the small capacitors C3--C11.  Pay attention to the placement of 
       C3, C8, C10, and C11 so that they don't interfere with the ribbon
       cables or CPU chip placement.
 \item Solder the ribbon cable connectors J1 and J2, and the programmer
       headers J5 and J6.
 \item Place the opto-isolator U1 on the board and solder into place, taking
       care to properly heat-sink the pins as you do so.
       The square pin on the PCB denotes the location of pin~1 of the chip.
 \item Install the LEDs D1--D8.  Note that the square pad on the PCB
       indicates the anode (+) and the flat key marked on the silk screen
	   (and presumably the flat edge of the LED itself) indicates the cathode
	   ($-$) of the LED.  The colors of the LEDs, and their board placement, are:
	   \begin{center}
	    \begin{tabular}{cccc}\\
		   amber & green & amber & red \\
		         & green \\
	             & green & amber & red 
		\end{tabular}
	   \end{center}
 \item Install the crystals X0 and X1 and solder
       into place.
 \item This completes the logic circuitry of the controller.  Now proceed to
       build the power supply on the left side of the board.
 \item Place the fuse holder onto the PCB and secure with a \#6 bolt,
       lockwasher and nut. Solder its terminals to the PCB.
 \item Place the terminal block J0 into position
       with the terminals facing out toward the edge 
       of the PCB, and solder into place.
 \item Place the rectifier bridge D0 onto the
       board, noting the indicated position of 
       the positive output lead.  Solder.
 \item Place capacitor C0 on the board, noting the
       position of the positive lead (``+'' marking
       and square pin).  Solder.
 \item Place the voltage regulator and its heat sink
       onto the board.  Be sure to apply heat sink grease between them to
	   ensure heat transfer.  
%	   Install the regulator with pin~1 in the hole
%	   designated with the square pad, and 
	   Secure the heat sink and regulator
	   with a \#6 bolt, lockwasher, and nut.  Solder the regulator pins, 
	   taking care to
	   heat sink them appropriately.
 \item If you are using a heat sink which is wide and tall enough that it
       may spin on the bolt and short on the regulator pins or other 
	   components, note that there are a pair of tiny holes provided to 
	   which you may
	   solder a short length of wire to secure the heat sink into place.
 \item Place and solder capacitors C1 and C2 near the voltage regulator.
 \item Place the transformer T0, taking care to 
       locate the primary leads on the side next
	   to the fuse, and the secondary leads on the
	   side next to the rectifier and regulator.
	   The primary leads are marked on the transformer as pins \#1, 3, 4,
	   and 6.  The secondary leads are \#7, 9, 10, and 12.  (On the
	   transformers I bought from Digi-Key, this means that the writing on
	   the transformer will be upside-down.  Your mileage may vary, so check.)
 \item Secure the transformer T0 to the PCB using
       two \#4 screws.  Run the screws down through
	   the mounting holes on the transformer, and on through the holes in the
	   PCB.  Place an \emph{insulated} standoff (or very thick washer) over 
	   each screw and secure with a
	   nut.  Be sure the metal mounting hardware is not threatening to short
	   against the transformer leads.  %In fact, it may be best to use an
	   %insulated standoff instead of a washer so the nut is well clear of the
	   %transformer's electrical terminals.  
	   Solder the transformer leads to the PCB.
 \item Insert a 1.25A fuse (F0) into the fuse holder.
 \item Use insulated stand-offs and appropriate machine bolts, washers and
       nuts to mount the controller PCB into its enclosure.
\end{enumerate}

\section{Connections}
This section describes how the various components of the assembled controller
system connect together.  

First of all, the controller board connects to each of the relay boards via
26-conductor ribbon cables.  These supply logic power and control signals
to the relay boards.  Make sure to route the ribbon cables out of the way of
the other parts, particularly the high-voltage areas.

Connecting main power to the controller board is done on terminal block J0.

%\begin{center}
% \begin{tabular}{cccc}
%  \multicolumn{2}{l}{120\,V AC} &
%  \multicolumn{2}{r}{Switch}\\
%  \hline
%  \multicolumn{1}{|c}{$\bigcirc$}&
%  $\bigcirc$&
%  $\bigcirc$&
%  \multicolumn{1}{c|}{$\bigcirc$}\\
%  \hline
% \end{tabular}
%\end{center}

If you wish to use a power switch, connect 120\,V AC across the left two
terminals (preferably with hot on the left, but that's not critical), and a
sufficiently rated switch across the right two terminals, to short those two
terminals when ``on''.

%\begin{center}
% \begin{tabular}{cccc}
%  \multicolumn{2}{l}{120\,V AC} &
%  \multicolumn{2}{r}{Switch}\\
%  \hline
%  \multicolumn{1}{|c}{$\bigcirc$}&
%  $\bigcirc$&
%  $\bigcirc$&
%  \multicolumn{1}{c|}{$\bigcirc$}\\
%  \hline
%  HOT&NEU&\multicolumn{2}{c}{(switch)}
% \end{tabular}
%\end{center}
\begin{center}
\epsffile{ac-switch.eps}
\end{center}

If you \emph{do not} wish to use a power switch, just connect the 120\,V AC
to the outer two terminals, leaving the center two terminals disconnected.

%\begin{center}
% \begin{tabular}{cccc}
%  \multicolumn{2}{l}{120\,V AC} &
%  \multicolumn{2}{r}{Switch}\\
%  \hline
%  \multicolumn{1}{|c}{$\bigcirc$}&
%  $\bigcirc$&
%  $\bigcirc$&
%  \multicolumn{1}{c|}{$\bigcirc$}\\
%  \hline
%  HOT&&&NEU
% \end{tabular}
%\end{center}
\begin{center}
\epsffile{ac-straight.eps}
\end{center}

The power connections on the relay boards are arranged into blocks of eight
circuits per block (for a total of six blocks for 48 channels).  Each block
is an independent, isolated unit (this allows greater control over
distributing power evenly rather than piling all 48 outputs on a single
power source).

The connections for each block appear on a pair of 10-position terminal
blocks.  For ``block~0'' on each relay board, that is terminals J1 and J2 
in the upper left of the board.  For ``block~1'' it is terminals J3 and J4,
which are on the right top and right bottom.  For ``block~2'' those
connections are on J5 and J6 in the lower left of the board.

These diagrams show the connections for block~0.  Note that the other blocks
are wired similarly, \emph{however, the locations of the connections vary
slightly.  Refer to the board markings to guide you.}

Each block has its own 120\,V AC input which supplies the load power for the
eight channels in that block.  In each case, the 120V input is split with the
hot wire connected on one terminal block, and the neutral wire on the other
block.  Refer to the board markings to see where each goes.  The terminals
adjacent to the power inputs are not connected to anything.

%\begin{center}
% \begin{tabular}{ccccccccccccccccccccc}
%   &&&&&&&&&\tiny N&\tiny 120V&\tiny H\\
%  \cline{1-10}\cline{12-21}
%   \multicolumn{1}{|c}{$\circ$}&
%   $\circ$&
%   $\circ$&
%   $\circ$&
%   $\circ$&
%   $\circ$&
%   $\circ$&
%   $\circ$&
%   $\circ$&
%   \multicolumn{1}{c|}{$\circ$}&
%   &
%   \multicolumn{1}{|c}{$\circ$}&
%   $\circ$&
%   $\circ$&
%   $\circ$&
%   $\circ$&
%   $\circ$&
%   $\circ$&
%   $\circ$&
%   $\circ$&
%   \multicolumn{1}{c|}{$\circ$}\\
%  \cline{1-10}\cline{12-21}
% \end{tabular}
%\end{center}
\begin{center}
\epsffile{relays.eps}
\end{center}

The outputs are arranged in twos, four per terminal block.  You just need to
strip enough bare wire on the power cords leading to your loads to insert
into the terminals corresponding to that channel.  Note how channel~1 plugs
in to a pair of adjacent terminals, and channel~0 plugs into the pair of 
terminals on either side of them.  

%\begin{center}
% \begin{tabular}{ccccccccccccccccccccc}
%   \tiny 0H&\tiny 1H&
%   \tiny 1N&\tiny 0N&
%   \tiny 3N&\tiny 2N&
%   \tiny 2H&\tiny 3H&&\tiny In N\\
%  \cline{1-10}
%   \multicolumn{1}{|c}{$\circ$}&
%   $\circ$&
%   $\circ$&
%   $\circ$&
%   $\circ$&
%   $\circ$&
%   $\circ$&
%   $\circ$&
%   $\circ$&
%   \multicolumn{1}{c|}{$\circ$}\\
%  \cline{1-10}
% \end{tabular}
%\end{center}

Finally, a communications line needs to be run from the controlling computer
to all the units it will control.

You will need an RS-485 output from your computer.  The most common way this
is accomplished is to get an RS-232 to RS-485 converter.  The converter plugs
into your PC's serial port and also to a length of CAT3 or CAT5 unshielded
twisted pair cable, wired as indicated above.  The other end of the CAT3
cable ends in an RJ-45 connector and plugs into the first controller's
``RS-485 input'' jack.  

If you have more than one unit, plug a straight-through cable into the
``RS-485 output'' jack of that controller, and into the ``RS-485 input'' jack
on the next unit, and so on for up to 16~controllers.  The last controller in
the chain should have a terminator plug installed into its ``RS-485 output''
jack.

You may connect more than one serial port, each with its own RS-485 converter
and up to 16~units.

\begin{center}
\epsffile{network.eps}
\end{center}

\section{Usage}
\subsection{Operation}
Operating the units is, from a hardware standpoint, fairly
straightforward.  You simply plug in the units as described
here, turn them on, and run your host software to drive them.

Since the actual experience of operating the controllers is
dependent upon, and centered around, the software itself,
please refer to your software documentation for more details.

\subsection{Communications Protocol}
The protocol used by the light controllers is a simple 
binary sequence of command and data bytes.  It is 
designed to be fast for transmitting data (by reducing the 
number of bytes sent per command) and fast to interpret
in the controller firmware.  It is also structured so 
that several completely different devices can share a
common serial bus, using different command sets (but
this same protocol format), without confusing each other.

In general, any given byte is either a ``command'' or 
``data'' byte.  If the most significant bit (MSB, or
bit~7) is set (1), then it is a command byte, with the
least significant four bits denoting the device address
and the other three bits holding the command to be
executed:
\begin{center}
\begin{bytefield}{8}
\bitheader[b]{0-7}\\
\bitbox{1}{1} &
\bitbox{3}{cmd} &
\bitbox{4}{addr} \\
\end{bytefield}
\end{center}

Data bytes always have their MSB clear (0) and can
contain any 7-bit data value:
\begin{center}
 \begin{bytefield}{8}
  \bitheader[b]{0-7}\\
  \bitbox{1}{0} & \bitbox{7}{data} \\
 \end{bytefield}
\end{center}

Given that basic structure, any device can simply ignore
any unexpected data byte, as well as any command byte 
not addressed to itself.  

If more than eight commands are needed, one command
pattern can be reserved to indicate that a follow-on 
data byte contains an extended command number.  I suggest
using command 7 ($111_2$) for this ``escape'' command:
\begin{center}
 \begin{bytefield}{8}
  \bitheader[b]{0-7}\\
  \bitbox{1}{1} & \bitbox{3}{7} & \bitbox{4}{addr} \\
  \bitbox{1}{0} & \bitbox{7}{command} \\
 \end{bytefield}
\end{center}

Using only this simple scheme allows for 135 commands.
Other schemes can be employed to get any arbitrary
command set desired.

This version of the 48-channel controller supports the
following command set:

% 0      All off
% 1 c    Channel c on/off
% 2 c v  Channel c to dimmer level v
% 3      Reserved
% 4      Reserved
% 5      Reserved
% 6      Reserved
% 7 x    Administrative command
%        x=00pppppp  set phase offset=p and reboot
%        x=010baaaa  set addr=a; b=a<1:0>
%        x=01100000  shutdown
%        x=01100001  disable admin command set
%
\subsubsection{0: All Channels Off}
\begin{center}
 \begin{bytefield}{8}
  \bitheader[b]{0-7}\\
  \bitbox{1}{1} & \bitbox{3}{0} & \bitbox{4}{addr}
 \end{bytefield}
\end{center}

This command causes all 48 channels to immediately turn
off and stay off until otherwise commanded.

The command byte for this would be (hex) {\tt\$8$a$}
i.e.\ the binary sequence {\tt 1000$aaaa$}.

\subsubsection{1: Set Channel On/Off}
\begin{center}
 \begin{bytefield}{8}
  \bitheader[b]{0-7}\\
  \bitbox{1}{1} & \bitbox{3}{1} & \bitbox{4}{addr}\\
  \bitbox{1}{0} & \bitbox{1}{s} & \bitbox{6}{channel}
 \end{bytefield}
\end{center}

This sets the status of the specified channel (0--63) to
be fully on (if $s=1$) or off ($s=0$).  Note that the protocol
supports 64 channels per device, while this model of controller
only has 48 channels.  This unit will ignore requests for 
channels \#48--63.

The command bytes for this would be (hex) {\tt\$9$axx$}
i.e.\ the binary sequence {\tt 1001$aaaa$ 0$scccccc$}.

\subsubsection{2: Set Channel Dimmer Level}
\begin{center}
 \begin{bytefield}{8}
  \bitheader[b]{0-7}\\
  \bitbox{1}{1} & \bitbox{3}{2} & \bitbox{4}{addr}\\
  \bitbox{1}{0} & \bitbox{1}{m} & \bitbox{6}{channel}\\
  \bitbox{1}{0} & \bitbox{7}{level}
 \end{bytefield}
\end{center}

For units capable of dimming, this command sets the dimmer level
for the specified channel (0--255).  256 levels of brightness are
supported in this firmware version, so 
a level of 0 is equivalent to turning
the channel completely off, and a level of 255 is equivalent to 
turning it completely on.  

If a particular model of Lumos SSR unit implements a lower dimmer
resolution (e.g., 128, 64, or 32 levels) then it will consider the
unused bits to be the \emph{least significant} part of the value.
This means the range is still 0--255, but the dimmer jumps in
larger increments between each level.  A 32-level dimmer would
consider levels 0--31 to be off, 32--63 to be the first dimmer level
up from that, etc.

The data bits in byte 3 are the most significant 7 bits of the value,
while the very least significant bit is shown as $m$ in byte 2 of
the sequence.  If this is simply set as $m=0$, the dimmer value
may be given as a number from 0--127.

%If more levels of dimmer resolution are
%supported in the future, then more bits of this byte will be 
%significant, making the value of {\it level\/} have a larger range
%of numbers between fully off and fully on.  (As it stands now, those
%upper bits are ignored, so if you sent a value of 32, the device
%would read that as 0; 33 as 1; 34 as 2; and so on.)

Note that the protocol 
supports 64 channels per device, while this model of controller
only has 48 channels.  This unit will ignore requests for 
channels \#48--63.

The command bytes for this would be (hex) {\tt\$A$axxyy$}
i.e.\ the binary sequence {\tt 1010$aaaa$ 00$cccccc$ 000$vvvvv$}.

\emph{Changed in version 3.0: Increased resolution from 64 to 256 dimmer
levels, introduced the $m$ high-resolution bit in the second byte.  Also
changed the behavior of how the controller responds if it implements
lower dimmer resolution than the protocol addresses.}

\subsubsection{3--6: Reserved Commands}
\begin{center}
 \begin{bytefield}{8}
  \bitheader[b]{0-7}\\
  \bitbox{1}{1} & \bitbox{3}{3--6} & \bitbox{4}{addr}\\
 \end{bytefield}
\end{center}

These four command codes are reserved to allow for future enhancements to have
single-byte commands available.

\subsubsection{7: Administrative Commands}
\begin{center}
 \begin{bytefield}{8}
  \bitheader[b]{0-7}\\
  \bitbox{1}{1} & \bitbox{3}{7} & \bitbox{4}{addr}\\
 \end{bytefield}
\end{center}

Since administrative commands are given when configuring devices, and not during an
actual show, we avoided giving them the smaller single-byte command codes.  Instead,
they are all sub-commands of command code \#7, which always has a follow-on data byte
which provides the additional information needed for the administrative function being
requested.

Note that these commands reconfigure the controller in significant ways.  The normal
processing of light control tasks (and other normal operations of the unit) may be
suspended while it is waiting for the follow-on byte(s) or while the command is
being executed.  It is possible that the unit will completely reboot after the
command in some cases, as noted below.  In these ways, the administrative commands
are quite unlike the normal light control commands.

\subsubsection{Admin: Set Device Address}
\begin{center}
 \begin{bytefield}{8}
  \bitheader[b]{0-7}\\
  \bitbox{1}{1} & \bitbox{3}{7} & \bitbox{4}{addr}\\
  \bitbox{1}{0} & \bitbox{1}{1} & \bitbox{1}{0} & \bitbox{1}{$z$} & \bitbox{4}{$y$}
 \end{bytefield}
\end{center}

This command changes the device address for the addressed unit.
Henceforth it will be known by the new address given in the data byte's
least-significant 4 bits ($y$).  This new address will be written to the device's 
EEPROM so the change is effectively permanent until this command is received
again at some future time.

The value $z$ consists of the least-significant bit of $y$.  Therefore, to
set the device address to \#10 (hex {\tt\$0A}), you would send the command bytes
(hex) {\tt\$F$a$4A}, i.e.\ the binary sequence {\tt 1111$aaaa$ 01001010}.

Upon completion of this command, the unit will display the new address on the LEDs
(just as it does when powering up) and will then resume normal operations with
the new address.  Bytes sent to it during that process may be partially or completely
lost.

\subsubsection{Admin: Shutdown Device}
\begin{center}
 \begin{bytefield}{8}
  \bitheader[b]{0-7}\\
  \bitbox{1}{1} & \bitbox{3}{7} & \bitbox{4}{addr}\\
  \bitbox{1}{0} & \bitbox{1}{1} & \bitbox{1}{1} & \bitbox{5}{0}
 \end{bytefield}
\end{center}

Upon receipt of this command, the unit shuts itself down to the least
power-consuming state possible, and halts, never responding to any further
input until it is fully reset or powered on again.  This is intended for
devices in embedded applications where there may be multiple units in a
single device enclosure, some of which are not needed for every application.
The extraneous devices can be shut down with this command.

The command bytes for this would be (hex) {\tt\$F$a$60}, i.e.\ the binary
sequence {\tt 1111$aaaa$ 01100000}.

\subsubsection{Admin: Set Phase Offset}
\begin{center}
 \begin{bytefield}{8}
  \bitheader[b]{0-7}\\
  \bitbox{1}{1} & \bitbox{3}{7} & \bitbox{4}{addr}\\
  \bitbox{1}{0} & \bitbox{1}{0} & \bitbox{6}{$p$}
 \end{bytefield}
\end{center}

This command changes the internal phase delay parameter to the value $p$.
This skews the AC line zero crossing point to compensate for any
anomolies which interfere with the dimmer timing of the unit.  Normally you
won't need to adjust this.

The effects of this command are permanent (until another adjustment is
requested).  The new phase offset is written to the controller's EEPROM.

Upon completion, the device is reset as if powered up; the new phase offset 
value is flashed on the status LEDs as part of the normal device initialization.
The value can range from 0--63, 
although values over about 40 would most likely not be useful.

The command bytes for this would be (hex) {\tt\$F$a$02} to set the phase offset to 2,
i.e.\ the binary sequence {\tt 1111$aaaa$ 00000010}.

\subsubsection{Admin: Disable Administrative Commands}
\begin{center}
 \begin{bytefield}{8}
  \bitheader[b]{0-7}\\
  \bitbox{1}{1} & \bitbox{3}{7} & \bitbox{4}{addr}\\
  \bitbox{1}{0} & \bitbox{1}{1} & \bitbox{1}{1} & \bitbox{5}{1}
 \end{bytefield}
\end{center}

Upon receipt of this command, the controller will refuse to honor any
subsequent administrative commands.  Any attempt to use one will cause
the lower red and yellow LEDs to turn on for 2~seconds, and the command
will be ignored.

This state remains in effect
until the unit is fully reset (by shorting the appropriate jumper pins or
by power-cycling the unit).  This prevents accidentally doing something 
bad such as re-addressing a unit while it's up and running.  (While there
is nothing technically wrong with re-addressing a unit on the fly, there's
plenty wrong with \emph{accidentally} doing so due to a communication 
glitch.)

When a controller is powered up (or reset), the yellow LEDs glow to indicate
(warn) that administrative functions are active.  Upon completion of this
command, those LEDs will extinguish.

The command bytes for this would be (hex) {\tt\$F$a$61},
i.e.\ the binary sequence {\tt 1111$aaaa$ 01100001}.


\subsection{Addressing}
It may be necessary to set the device address of each unit you will use
in a particular application.  Each unit in a single RS-485 network is
assigned an address value of 0--15.  No two units should ever have the
same address on the same network.  (In this version, it's actually not 
a problem, since the units never transmit data back to the host computer.
However, if a future enhanced version of the firmware includes response
data back to the host, having two identically-addressed units would cause
bus contention and data collisions as they all try to respond at the same
time.)

Note that as of version 3.0 of the firmware, address 15 (binary 1111) is
reserved as a ``broadcast'' address.  All units will respond to commands
addressed to this unit number.  It is also allowed to set a device to address
15, in which case it will only respond to the broadcast commands.

To change a unit's address, plug it into a computer (with no other units 
present on the RS-485 network) and power it up.  Watch the status lights 
blink as it intializes.  Watch for the following sequence of lights:
\begin{enumerate}
 \item Green LED on
 \item Red LED flashes 0--15 times to indicate unit \#0--15, respectively
 \item Green LED off
\end{enumerate}

The flashing amber LED will tell you what address the unit is currently
set to.  Now have the host computer send the address change command to
that device address, and have that command set the unit to the desired
new address.

Once the address is changed, the unit will reset as if power-cycled,
so you can watch the flashing LEDs to see that it is now set to
the new address.

%
% SLW: A design change just obsoleted this advice.  The slave firmware
% doesn't need to know the device address because the master will always
% send it only those commands already addressed to this unit.
%
%{\bf N.B.:}
%If the upper and lower status lights show a \emph{different} address from 
%one another, \emph{this is a serious problem and must be corrected immediately.}
%Here's the sequence you need to give:
%\begin{enumerate}
% \item Change the unit address to match the \emph{second} (``slave'' or lower)
%       set of status LEDs.
% \item Change the unit address to the actual desired new address.
% \item Verify that \emph{both} sets of LEDs show the new address now.
%\end{enumerate}

\subsection{Driver Software}
I have created a simple suite of platform-independent driver software 
components to send commands and orchestrate shows for these boards from
a host computer running Unix, MacOS X or Windows.  It is still under
development but notes about it will be placed here when it's further
along.

\section{Reprogramming}
If a firmware upgrade becomes available, you can transfer it into the CPUs
on the controller board using a compatible PIC microcontroller programmer.
The controller board is designed to accommodate in-circuit serial pro-
gramming (ICSP) using high-voltage programming mode.  Header J5 provides
access to the ``master'' CPU U2, and header J6 provides access to the ``slave''
CPU U3.  Be sure to put the correct firmware image on each chip (there is
a difference between the master program and the slave program, and a different
program for 16F777 and 16F877A chips).

The pinout of J5 and J6, as viewed from the top of the board, is:
\begin{center}
 \begin{tabular}{cccccc}
  5&4&3&2&1\\
  \cline{1-5}
  \multicolumn{1}{|c}{$\bigcirc$}&
  $\bigcirc$&
  $\bigcirc$&
  $\bigcirc$&
  \multicolumn{1}{c|}{$\bigcirc$}&
  \large$\triangleleft$\\
  \cline{1-5}
  PG$\phi$&PGD&V$_{\hbox{\small PP}}$&V$_{\hbox{\small SS}}$&V$_{\hbox{\small
  DD}}$\\
 \end{tabular}
\end{center}

\noindent These signals are:
\begin{description}
\item[PG$\phi$] (Pin~5) Program clock signal.  The programmer uses
synchronous serial communication, with a pulse on this pin indicating
that the {\bf PGD} signal is ready to be sampled for the next bit.
\item[PGD] (Pin~4) Program data.  The programmer's commands to the CPU are
sent serially on this pin, clocked by the {\bf PG$\phi$} signal.
\item[V$_{\hbox{\small PP}}$] (Pin~3) Program voltage.  A high voltage on
this pin (+13.5V) places the CPU into program mode.  Note that this does not
actively \emph{power} the programming, it just signals it; so there is almost
no current drain (a few $\mu$A, actually) on this pin.
\item[V$_{\hbox{\small SS}}$] (Pin~2) Ground.
\item[V$_{\hbox{\small DD}}$] (Pin~1) +5V supply.  A low-power programmer may
actually be powered from pins 1 and 2, or it can ignore pin~1 and simply use
pin~2 as a signal ground reference.
\end{description}

Note the triangle printed on the circuit board (and the square PCB pad) to denote pin~1.
\end{document}


\section{Appendix A: Partial Construction Options}
There are a few options for alternate construction which allow you to
build less than the full circuit.  These will be described here, in 
case they are of use in some circumstances.

\subsection{Using an External Power Supply}
If you look at the controller board (part ``48CTL-$x$-$x$''), you will
note that approximately half the board is the actual controller circuit,
and the other half is merely the 5V supply which powers the controller
and the logic portions of the relay boards.

Since the controller board needs to monitor the AC line waveform anyway,
this way you can just supply a single 120\,V AC source to the board, and
that will take care of both functions without needing an external power
supply.  However, if you don't wish to build that half of the board,
or can't locate the transformer specified, or some other issue prevents
you from building it, you can supply the logic power from another source.

You have several options:
\begin{enumerate}
\item 
	If you have an unregulated or regulated +9V supply (AC or DC), you 
	can omit transformer T0 and wire in the power supply to the terminals
	where the transformer's secondary coils would have been (either the 
	pair on the left or the pair on the right, or the inside pair, or the
	outside pair).
\item
	Or, if the +9V supply is DC, you can leave out D0 and connect +9V
	to the (+) pin where D0 would have been, and ground to the pin 
	on the other far end of D0's position (leaving the two middle
	terminals unconnected).
\item
	Or, if you have a \emph{regulated} +5VDC supply, you can omit 
	C0, C1, C2, D0, T0, and U0.  Connect the supply output to the
	terminals where C2 would have gone, with the +5V lead in the 
	bottom hole and the ground lead in the top hole.  (The ``bottom''
	of the board is the edge with the RJ-45 jacks.)
\end{enumerate}

The power supply needs to be able to supply a sustained 800mA draw; a
rating of 1,000mA or better is recommended.

Please note that regardless of these options, you \emph{still} need to
supply 120\,V AC on terminal block J0, and provide components F0, R1, R2,
U1 and R14.

\subsection{Building Fewer Than 48 Channels}
You may build the controller to drive anywhere from 1 to 48 channels of 
120\,V AC loads.  

The best approach is to build complete controller and relay boards, but
if for some reason you need to build the absolute minimum possible,
you have a few options to limit the scope of your construction:

If you want to use less than 25 channels, you only need one relay board and 
one controller board.  In this case, you can omit parts C6, C7, C8, C11, D6, 
D7, D8, J2, J6, R9, R10, R11, R12, R13, U3, X1, and connect U3 pin 7 to 
ground on the controller board.  However, I really wouldn't recommend doing 
this, because: (a) when you decide to go beyond 24 channels, you'll need to
add those parts anyway, (b) those parts are fairly inexpensive compared to
the overall cost of the project, and (c) leaving out those components makes
it impossible for the unit to transmit data back to the host computer.  But
if you just can't stand the idea of a CPU sitting idle all the time, or are
very tight on your budget, that is an option. (Note: The current version of 
the firmware does \emph{not} ever send data back, but it is implemented in the 
controller's hardware for future use, so it would theoretically only 
require a firmware upgrade to take advantage of when/if that becomes a 
new feature later.)

All the LEDs, and the resistors which supply them, are optional but 
recommended.

Looking at each relay board, it provides 24 individual output channels in
blocks of 8 channels.  Each block of 8 channels is separately powered and
fuse-protected, allowing more current to be available than my previous
design (8A total at any 
given time for any block of 8, up to 5A for any individual channel).  

You must provide D0, J0, R0 on the relay board (well, D0 and R0 are optional
as they merely provide a power light to verify that the board is plugged in
and getting power from the controller).  
Then for each 8-channel block you wish to use, add the
fuse, pair of terminal blocks, and the number of SSRs you wish to provide.
Each SSR is composed of parts R$x$0, R$x$1, TR$x$, and U$x$, where $x$ is the channel
number (0--23).  Leaving the entire set of (4) components off the board for
any given channel is ok if you don't want that channel implemented.

Or, better yet, just build the whole darn thing.
%
% Design notes:
%   Next rev, it might be better to add screw mounting holes
%   for the fuse holders.  I think the current design is sufficient,
%   but this would improve it.
%
%   Also, need to make the holes for 1/2W resistors bigger.
%

\section{Control Buttons}
The DC relay boards include two on-board pushbuttons (when the board is configured for stand-alone mode with a microcontroller present) to control and configure the board manually.

The older model 48-chanel controller board can be retro-fitted with pushbuttons if desired.  (See the diagram in figure XXX for details.)

When ``reset'' button is pressed, the microcontroller chip is halted and reset.  When the button is released, the unit begins operation as if just powered on.  This can be used if the unit becomes unresponsive for some reason, or communication has been lost with the controlling PC and you need a fast way to stop all output and reset the unit.

The ``option'' button is used to set various options manually on the unit and to go into various test modes.  To begin manual option setting, press and hold the ``option'' button until the red status light flashes rapidly.  All other operation of the unit will cease until the option settings have been completed.  All serial input will be ignored and all channel outputs will be turned off.

While in option mode (red light flashing rapidly), push and release the option button to cycle between following modes.  When the desired mode is selected, press and hold the button until the red light flashes slowly.  This will activate that mode.  Pushing and holding the option button again will terminate that operation and exit option mode, resuming normal operation again.

\begin{enumerate}
	\item \strong{Channel Test Cycle [--{}--YR]}
		This mode is indicated with the yellow light on steady, and green and activity lights off.  When activated,
		When this mode is enabled, Each channel in turn from 0--47 is turned on for one second.  Pressing the option button
		briefly will pause the cycle (so the currently active channel remains on, if any).  Pressing it again resumes the cycle.  After the
		last channel is done, a one-second ``off'' cycle takes place before channel 0 turns on again.
	\item \strong{Address Change [--G--R]}
		This mode is indicated with the green light on steady, and the yellow and activity lights off.  When activated,
		the unit displays its current address by temporariy blanking out all four LEDs and lighting the unit address
		in binary on them:
		\begin{tabular}{|cccc|r|}\hline
			act & grn & yel & red & address \\\hline\hline
			off & off & off & off & 0 \\
			off & off & off & on & 1 \\
			off & off & on & off & 2 \\
			\multicolumn{4}{|c}{$\vdots$} & $\vdots$ \\
			on & on & on & on & 15\\\hline
		\end{tabular}
		Each time the option button is pressed, the address increments by one, as reflected in the LED
		readout.  Push and hold the button to exit this mode, keeping the address currently displayed.
	\item \strong{Phase Offset Change [--GYR]}
		\emph{Use caution when changing this value; improper setting may prevent the AC relays from
		functioning properly when dimming loads.}
		This mode is indicated with the green and yellow lights on steady, activity light off.  When activated,
		the unit displays the most significant nybble of the phase offset (the default value is 0) by blanking the 
		LEDs, flashing the activity light, then displaying the nybble on the four LEDs in binary.  Pressing the button
		will advance the nybble by one.  After 5 seconds of no button presses, the red light will flash, then the
		least significant nybble will be displayed and may be changed in the same manner.  After 5 seconds,
		the yellow light will flash, and you will have a four-second window in which to push and hold the button
		to save this change and exit.  Otherwise the change is canceled. (This is intended as a guard against
		accidentally changing this setting.)
	\item \strong{Enable Administrative Commands [AGYR]}
		This mode is indicated with all lights lit.
		Ordinarily, the administrative commands are disabled for the serial command interface.  
		When this mode is activated, the green LED is lit if administrative commands are enabled,
		or the yellow LED is lit if they are disabled.  Pressing the button toggles between the two.
		This affects the ``Set device address'' and ``Set Phase Offset'' commands.  It is still possible
		to issue device shutdown commands in any case.
\end{enumerate}
